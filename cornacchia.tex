\documentclass[12pt,leqno]{article}

\usepackage[a4paper,margin=1in,landscape]{geometry}
\usepackage[utf8]{inputenc}
\usepackage[T1]{fontenc}
\usepackage{parcolumns}
\usepackage[italian,american]{babel}
\usepackage{amsmath,amssymb,amsthm}
\usepackage{fmtcount}

\newtheorem{coroen}{Corollary}
\newtheorem{coroit}{Corollario}
\newenvironment{corollary}
{\iflanguage{italian}{\coroit}{\coroen}}
{\iflanguage{italian}{\endcoroit}{\endcoroen}}

\newcommand{\cols}[2]{\begin{parcolumns}[rulebetween]{2}%
    \selectlanguage{italian}\colchunk{#1}%
    \selectlanguage{american}\colchunk{#2}%
  \end{parcolumns}}
\newcommand{\colsni}[2]{\begin{parcolumns}[rulebetween,nofirstindent]{2}%
    \selectlanguage{italian}\colchunk{#1}%
    \selectlanguage{american}\colchunk{#2}%
  \end{parcolumns}}

\newcommand{\footnotecols}[2]{\footnotetext{%
    \begin{minipage}[t]{0.95\linewidth}\vspace{-1em}\cols{#1}{#2}\end{minipage}%
}}

\newcommand\secbreak{%
  \par\bigskip\noindent\hfill\rule{15em}{0.1px}\hfill\null\par\bigskip
}

\title{\selectlanguage{italian}Su di un metodo per la risoluzione in
  numeri interi
  dell'equazione\\
  $\sum_{h=0}^nC_hx^{n-h}\cdot y^h=P$}
\author{Dott. Giuseppe Cornacchia\\[1em]
  translation by Luca De Feo}

\begin{document}
\maketitle

\cols{Il metodo che andiamo ad esporre si fonda sul noto
  \emph{algoritmo di Eulero} che si applica alla risoluzione
  dell'equazione $ax+by=1$.
  
  Se si hanno due quantità indeterminate qualunque $a,b,$ e una serie
  di altre quantità $\gamma,\delta,\varepsilon,\ldots,\lambda,\mu,\nu$,
  e si forma con esse una nuova serie di quantità $c,d,e,\ldots,l,m,n$
  operando secondo la legge seguente:}
{The method we're going to explain is funded on Euler's
  well known algorithm, which applies to the solution of the
  equation $ax+by=1$.
  
  If we have two arbitrary indeterminate quantities $a,b,$ and a
  series of other quantities
  $\gamma,\delta,\varepsilon,\ldots,\lambda,\mu,\nu$, and we form with
  them a new series of quantities $c,d,e,\ldots,l,m,n$ acting by the
  following law:}

\[c = \gamma b + a\quad
  d = \delta c + b\quad
  e = \varepsilon d + e,\ldots,
  n = \nu m + l\]

\colsni{si potranno ottenere tutte le quantità $c,d,e,\ldots,l,m,n,$
  successivamente, in funzione di $a$ e $b$. In particolare:}
%
{we can obtain all quantities  $c,d,e,\ldots,l,m,n,$
  successively, as a function of $a$ e $b$. In particular:}

\[n = Ga+Hb\]

\colsni{dove $G,H$ sono funzioni di $\gamma,\delta,\varepsilon,\ldots$
  indipendenti da $a$ e $b$. Usando il simbolo di Gauss, si ha}
%
{where $G,H,$ are function of $\gamma,\delta,\varepsilon,\ldots$
  independent from $a$ and $b$. Using Gauss' symbol, we have}

\[H = [\gamma,\delta,\varepsilon,\ldots,\lambda,\mu,\nu]\quad
  G=[\delta,\varepsilon,\ldots,\lambda,\mu,\nu].\]

\cols{Le espressioni $H$ obbediscono a due semplici leggi di
  formazione, l'una procedendo verso destra, l'altra verso sinistra,
  che sono espresse dalle formule:}
%
{The expressions $H$ obey two simple laws, one proceding from the
  right, the other from the left, as expressed by the formulas:}
%
\begin{align*}
  [\gamma,\delta,\varepsilon,\ldots,\lambda,\mu,\nu] &=
  \gamma[\delta,\varepsilon,\ldots,\lambda,\mu,\nu] +
  [\varepsilon,\ldots,\lambda,\mu,\nu]\\
  [\gamma,\delta,\varepsilon,\ldots,\lambda,\mu,\nu] &=
  [\gamma,\delta,\varepsilon,\ldots,\lambda,\mu]\nu +
  [\gamma,\delta,\varepsilon,\ldots,\lambda].
\end{align*}

\cols{Si ha poi}{Then we have}

\[[\gamma,\delta,\varepsilon,\ldots,\lambda,\mu,\nu] =
  [\nu,\mu,\lambda,\ldots,\varepsilon,\delta,\gamma];\quad
  [-\gamma,-\delta,-\varepsilon,\ldots,-\lambda,-\mu,-\nu] =
  \pm[\gamma,\delta,\varepsilon,\ldots,\lambda,\mu,\nu].\]

\cols{Queste espressioni sono della massima importanza nella teoria
  delle frazioni continue. Infatti, se una frazione continua
  ordinaria, i cui numeratori sono tutti uguali ad 1, si rappresenta
  col simbolo
  $(\gamma,\delta,\varepsilon,\ldots,\lambda,\mu,\nu,\ldots)$ talché
  sia}
%
{}

\[(\gamma,\delta,\varepsilon,\ldots,\lambda,\mu,\nu) =
  \gamma+\frac{1}{(\gamma,\delta,\varepsilon,\ldots,\lambda,\mu,\nu)} =
  \left(\gamma,\delta,\varepsilon,\ldots,\lambda,\mu+\frac{1}{\nu}\right)\]

\colsni{si ottiene in generale, mediante riduzione:}{}

\[(\gamma,\delta,\varepsilon,\ldots,\lambda,\mu,\nu) =
  \frac{[\gamma,\delta,\varepsilon,\ldots,\lambda,\mu,\nu]}
  {[\delta,\varepsilon,\ldots,\lambda,\mu,\nu]}.\]

\cols{Se i numeri $\gamma,\delta,\varepsilon,\ldots,\lambda,\mu,\nu$
  sono interi, altrettanto può dirsi dei numeratori e dei denominatori
  delle frazioni}
%
{}

\[\frac{[\gamma]}{[1]}, \frac{[\gamma,\delta]}{[\delta]},
  \frac{[\gamma,\delta,\varepsilon]}{[\gamma,\delta]}, \ldots,
  \frac{[\gamma,\delta,\varepsilon,\ldots,\lambda,\mu,\nu]}
  {[\delta,\ldots,\lambda,\mu,\nu]},\]

\colsni{ognuna delle quali è inoltre irreducibile.

  Vediamo qualche altra propriétà delle espressioni sopra
  accennate.  --- Se, prese delle quantità $1,q_1,q_2,\ldots,q_{n-1},q_n$
  si formano successivamente le espressioni:}
%
{}
%
\begin{gather*}
  y_{n+1}=1\quad y_n=1\quad y_{n-1}=[1,q_1]\quad y_{n-2}=[1,q_1,q_2),\ldots\\
  y_1=[1,q_1,q_3,\ldots,q_{n-1}]\quad;\quad
  y_0=[1,q_1,q_2,\ldots,q_{n-1},q_n]
\end{gather*}

\colsni{e poi le epressioni}{}
%
\begin{gather*}
  x_{n+1}=1\quad x_n=1\quad x_{n-1}=[1,q_n]\quad x_{n-2}=[1,q_n,q_{n-1}],\ldots\\
  x_1=[1,q_n,q_{n-1},\ldots,q_2]\quad;\quad
  x_0=[1,q_n,q_{n-1},\ldots,q_2,q_1]  
\end{gather*}

\colsni{dico che si ha:}{}
%
\begin{align}
  &x_r\cdot y_{n-r} + x_{r+1}\cdot y_{n-r+1} = P
  &(r=0,1,2,\dots,n)
  \tag{\(\alpha\)}\label{alpha}\\
  &x_r\cdot y_s + (-1)^{n+r+s}\cdot x_{n-s+1}\cdot y_{n-r+1} \equiv 0 \mod P
  &\left(\begin{array}{c}r=0,1,\dots,n\\s=0,1,\dots,n\end{array}\right)
  \tag{\(\beta\)}\label{beta}
\end{align}

\colsni{avendo posto $P=x_0+x_1=y_0+y_1$. La \eqref{beta} vale
  naturalmente solo quando le $q$ sono numeri interi.  --- Per la legge
  esposta di formazione, si ha:}
%
{}

\[x_{r+1} = x_{r-1} - q_r x_r \quad y_{n-r}=q_r y_{n-r+1} + y_{n-r+2}\]

\colsni{posto}{}

\[q_0 = 1 \quad x_{-1} = x_0 + x_1 = P.\]

\cols{Sostituendo nel \ordinalnum{1} membro della \eqref{alpha},
  avremo:}{}

\[x_r\cdot y_{n-r} + x_{r+1}\cdot y_{n-r+1} =
  x_r(q_r y_{n-r+1} + y_{n-r+1}) + y_{n-r+1}(x_{r-1} - q_r x_r) =
  x_{r-1}\cdot y_{n-r+1} + x_r\cdot y_{n-r+2}\]

\cols{Dunque l'espressione $x_r\cdot y_{n-r} + x_{r+1}\cdot y_{n-r+1}$
  non cambia valore mutando $r$ in $r-1$, e quindi neppure cambiando
  $r$ in $r-\lambda (\lambda=0,1,2,\dots r)$.  Laonde:}
%
{}

\[x_r\cdot y_{n-r} + x_{r+1}\cdot y_{n-r+1} =
  x_{r-\lambda}\cdot y_{n-(r-\lambda)} + x_{(r-\lambda)+1}\cdot y_{n-(r-\lambda)+1}.\]

\colsni{e per $r=\lambda$ si ha}{}

\[x_r\cdot y_{n-r} + x_{r+1}\cdot y_{n-r+1} = x_0 + x_1 = y_0 + y_1 = P.\]

\cols{Così è dimostrata la \eqref{alpha}; dalla \eqref{beta} si possono
  come casi particolari estrarre le due congruenze}
%
{}
%
\begin{align}
  x_r + (-1)^rx_1\cdot y_{n-r+1} &\equiv 0 \mod P
  \tag{\(\beta_1\)}\label{beta1}\\
  y_s + (-1)^s\cdot y_1 \cdot x_{n-s+1} &\equiv 0 \mod P
  \tag{\(\beta_2\)}\label{beta2}
\end{align}
 
\colsni{le quali intanto sussistono, come si verifica subito, la
  \ordinalnum{1}[f] per $r=0,1$; la \ordinalnum{2}[f] per
  $s=0,1$.  --- E col solito metodo da $r$ a $r+1$ (o da $s$ ad $s+1$)
  valendosi delle relazioni che legano le quantità $x$ e le $y$, si
  dimostra che sono vere per qualunque valore degli indici $r,s$
  rispettivamente. ---  Le \eqref{beta1}, \eqref{beta2} si possono anche
  scrivere:}
%
{}
%
\begin{align}
  x_r &\equiv - (-1)^rx_1\cdot y_{n-r-1} \mod P
  \tag{\(\beta_1'\)}\label{beta1'}\\
  y_s &\equiv - (-1)^sy_1\cdot x_{n-s-1} \mod P
  \tag{\(\beta_2'\)}\label{beta2'}
\end{align}

\colsni{e moltiplicando membro a membro}{}

\[x_r\cdot y_s \equiv (-1)^{r+s}\cdot x_1\cdot y_1\cdot y_{n-r+1}\cdot x_{n-s+1}.\]

\cols{Ma la \ref{beta1'} dà per $r=n$}{}

\[1 \equiv - (-1)^n x_1 \cdot y_1\]

\colsni{e però, come si voleva:}

\[x_r\cdot y_s + (-1)^{n+r+s}\cdot x_{n-s+1}\cdot y_{n-r+1} \equiv 0 \mod P.\]

\cols{Ciò posto, sia $P$ un intero qualunque, ed $a$ un numero primo
  con $P$ e minore di $P$ ma maggiore di $\frac{1}{2}P$. Si applichi
  alla coppia $(P,a)$ l'algoritmo della ricerca del M.C.D.; per
  l'ipotesi posta si perverrà al resto $1$ dopo un numero finito $n$
  di divisioni. Sia $q_0=1,q_1,q_2,\ldots,q_{n-1}$ la serie dei
  quozienti; $x_1,x_2,\ldots,x_n=1$ la serie dei resti corrispondenti.

  Posto $P=x_{-1},a=x_0,x_{n+1}=1,q_n=x_{n-1}-1$ avremo:}
%
{}

\begin{equation}
  \label{1}
  x_{i-1} = q_i x_i + x_{i+1}.\qquad
  (i=0,1,\ldots, n)
\end{equation}

\cols{Si ponga ora il sistema}{}

\begin{equation}
  \label{2}
  y_{n-i} = q_i y_{n-i+1} + y_{n-i+2}\qquad
  (i=0,1,\ldots, n)
\end{equation}

\colsni{a cui si aggiungano le condizioni $y_n=y_{n+1}=1$.

  Si avrà allora, introducendo il simbolo di Gauss:}

\[y_{n-i} = [1,q_1,q_2,\ldots,q_i]\qquad
  x_{i-1} = [1,q_n,q_{n-1},\ldots,q_i].\]

\cols{Richiamando le propriétà sopra dimostrate, avremo:}{}

\begin{align}
  &x_r\cdot y_{n-r} + x_{r+1}\cdot y_{n-r+1} = x_0 + x_1 = y_0 + y_1 = P
  &(r=0,1,2,\dots,n)
  \tag{\ref{alpha}}\\
  &x_r\cdot y_s + (-1)^{n+r+s}\cdot x_{n-s+1}\cdot y_{n-r+1} \equiv 0 \mod P
  &\left(\begin{array}{c}r=0,1,\dots,n\\s=0,1,\dots,n\end{array}\right)
  \tag{\ref{beta}}
\end{align}

\colsni{da cui in particolare per $r=s=1$}{}

\[x_1\cdot y_1 \equiv (-1)^{n-1} \mod P\]

\colsni{e ancora}{and}

\[x_0\cdot y_0 \equiv (-1)^{n-1} \mod P.\]

\cols{Si è trovata una soluzione $y=y$ della congruenza}{}

\[ay \equiv \pm 1 \mod P.\]

\cols{Ponendo mente alle relazioni ricorrenti~\eqref{2} e ricordando
  che $y_0+y_1=P$, si scorge come i numeri $y_1,y_2,\dots,y_n=1$ sono
  i resti delle successive divisioni che si ottengono applicando ai
  numeri $(P,y_0)$ l'algoritmo della ricerca del M.C.D. Si ha poi
  $x_0>\frac{1}{2}P,y_0>\frac{1}{2}P$. Se $n$ è dispari, allora
  $x_0 y_0 \equiv 1\mod P$, cioè la radice della congruenza}
%
{}

\[x_0 y \equiv 1 \mod P\]

\colsni{è maggiore di $\frac{1}{2}P$; se $n$ è pari, si ha
  $x_0y\equiv -1, x_0 y_1 \equiv 1 \mod P$; in questo caso la radice
  $y_1$ della congruenza $x_0y\equiv 1 \mod P$ è $<\frac{1}{2}P$. Dopo
  ciò possiamo enunciare:

  \guillemotleft{} Dato un intero qualunque $P$ e un numero
  $x_0<\frac{1}{2}P$, primo con $P$ e inferiore ad esso, sia $y_0$ la
  radice della congruenza}
%
{}

\[x_0 y \equiv 1\mod P\]

\colsni{in altri termini, sia $y_0$ il \emph{coniugato} di
  $x_0\mod P$: applicando ai numeri $(P,x_0)$ l'algoritmo della
  ricerca del M.C.D. si perviene al resto $1$ dopo un numero $n$ di
  divisioni pari o dispari, secondo che
  $y_0\lessgtr\frac{1}{2}P$.  --- In ogni caso, applicando lo stesso
  procedimento ai numeri $(P,y_0)$
  $\left\{ \text{oppure } (P,P-y_0) \text{ se } y_0<\frac{1}{2}P
  \right\}$ si perviene al resto $1$ dopo lo stesso numero $n$ di
  divisioni. ---  Le due successioni dei resti e dei quotienti godono le
  seguenti proprietà:

  Nei due sistemi di divisioni il \ordinalnum{2} quoziente dell'un
  sistema è uguale al penultimo resto dell'altro, diminuito di 1.  --- I
  quozienti della \ordinalnum{1}[f] successione, a cominciare dal
  terzo, si riproducono nella \ordinalnum{2}[f] in ordine inverso.

  La somma dei prodotti in croce di due resti consecutivi del
  \ordinalnum{1} sistema, e dei corrispondenti nel \ordinalnum{2}
  occupanti i posti complementari ad $n+1$ (cioè se i primi due resti
  occupano i posti $t,t+1$, i corrispondenti occuperanno i posti
  $n-t,n-t+1$) è uguale a $P$.

  Presi due resti qualunque $x_r,y_s$, l'uno del \ordinalnum{1}
  sistema, l'altro del \ordinalnum{2}, e i loro rispettivi
  complementari $x_{n-s+1},y_{n-r+1}$, si ha}
%
{}

\[x_r\cdot y_s + (-1)^{n+r+s}\cdot x_{n-s+1}\cdot y_{n-r+1}\equiv 0 \mod P.\]

\cols{In particolare, presi due resti $x_r,x_{n-r+1}$ del
  \ordinalnum{1} sistema equidistanti dagli estremi, e i
  corrispondenti $y_r,y_{n-r+1}$ nel \ordinalnum{2}, si ha:}
%
{}

\[x_r\cdot y_r + (-1)^n x_{n-r+1}\cdot y_{n-r+1}\equiv 0 \mod P.\]

\cols{Vediamo ora due casi particolari notevoli. Supponiamo in primo
  luogo che si abbia}
%
{}

\[x_0^2 \equiv 1 \mod P\]

\colsni{cioé sia $x_0$ coniugato di se stesso.  --- Allora $y_0=x_0$; e
  poiché $x_0 y_0 \equiv (-1)^{n-1} \mod P$, se ne deduce che dovrà
  essere $n$ dispari. Intanto possiamo dunque dire che se
  $x_0>\frac{1}{2}P$ è coniugato di se stesso $\mod P$, applicando ai
  numeri $P,x_0$ l'algoritmo della ricerca del M.C.D., si perviene al
  resto $1$ dopo un numero dispari $n$ di divisioni.  --- Siccome poi
  abbiamo trovato che i numeri $y_1,y_2,\ldots,y_{n-1},y_n=1$ sono i
  resti delle divisioni che si ottengono applicando l'algoritmo della
  ricerca del M.C.D. ai numeri $P,y_0$, ne consegue, essendo qui
  $y_0=x_0$, che i resti $y$ coincidono coi corrispondenti $x$ del
  sistema di divisioni $(P,x_0)$. ---  La relazione~\eqref{alpha}
  diventa allora:}
%
{}

\begin{equation}
  \tag{\(\alpha'\)}\label{alpha'}
  x_r\cdot x_{n-r} + x_{r+1}\cdot x_{n-r+1} = P
\end{equation}

\colsni{e la~\eqref{beta}}{}

\begin{equation}
  \tag{\(\beta'\)}\label{beta'}
  x_r\cdot x_s - (-1)^{r+s}\cdot x_{n-r+1}\cdot x_{n-s+1} \equiv 0 \mod P
\end{equation}

\colsni{dalla quale, per $r=s$ si trae l'altra:}

\begin{equation}
  \tag{\(\gamma'\)}\label{gamma'}
  x_r^2 - x_{n-r+1}^2 \equiv 0 \mod P.
\end{equation}

\cols{Il resto intermedio del sistema sarà $x_\frac{n+1}{2}$. Ponendo
  nella~\eqref{alpha'} $r=\frac{n+1}{2}$, si avrà:}
% 
{}

\begin{equation}
  \tag{\(\delta'\)}\label{delta'}
  x_{\frac{n+1}{2}}\left(x_{\frac{n-1}{2}} + x_{\frac{n+3}{2}}\right) = P
\end{equation}

\colsni{dunque il resto intermedio è un divisore di $P$.

  Le relazioni $x_i=y_i$ insieme
  alle~\eqref{alpha'},~\eqref{beta'},~\eqref{gamma'},~\eqref{delta'}
  che ne abbiamo dedotte, si traducono nelle seguenti proprietà:

  \guillemotleft{} Dato un numero $P$, sia $x_0>\frac{P}{2}$ una
  radice della congruenza}
% 
{}

\[x^2\equiv 1 \mod P.\]

\cols{Applicando ai numeri $P,x_0$ l'algoritmo della ricerca del
  M.C.D., la successione dei quozienti e quella dei resti godono delle
  seguenti proprietà:}
% 
{}

\cols{Il numero delle divisioni che bisogna eseguire per giungere al
  resto $1$ è sempre dispari.

  Il \ordinalnum{2} quoziente è uguale al penultimo resto diminuito di $1$.

  Fatta astrazione dei due primi quozienti, nella successione dei
  restanti i quozienti equidistanti dagli estremi sono eguali.

  La somma dei prodotti in croce di due resti consecutivi qualunque, e
  dei corrispondenti equidistanti dagli estremi, è uguale a $P$.

  Il resto intermedio è un divisore di $P$.

  La somma o differenza dei prodotti di due resti qualunque, e dei
  corrispondenti equidistanti dagli estremi, è multipla di $P$.

  Una coppia qualunque di resti equidistanti dagli estremi soddisfa
  alla congruenza}
% 
{}

\[x^2-y^2\equiv 0 \mod P.\text{ \guillemotright}\]

\cols{Adunque la conseguenza dei divisori di $P$ e delle soluzioni
  della congruenza $x^2-y^2\equiv 0 \mod P$ dipende dalla conoscenza
  delle radici della congruenza}
% 
{}

\[x^2\equiv 1 \mod P.\]

\cols{ --- Come secondo caso notevole, supponiamo che $x_0$ soddisfi
  alla congruenza}
% 
{}

\[x_0^3 + 1 \equiv 0 \mod P\]

\colsni{allora}{then}

\[x_0^2 \equiv -1 \mod P.\]

\cols{Ma è pure}{}

\[x_0 y_0\equiv (-1)^{n-1} \mod P\quad \left(
    \begin{array}{c}
      P > x_0 > \frac{P}{2}\\
      P > y_0 > \frac{P}{2}
    \end{array}
  \right)\]

\colsni{se ne trae che dovrá essere $n$ pari ed $y_0=x_0$.

  Per quanto abbiamo osservato, le $y_1,y_2,\ldots,y_{n-1},y_n=1$
  coincidono con le corrispondenti $x_1,x_2,\ldots,x_n=1$; poichè qui
  $(P,x_0)=(P,y_0)$, simboleggiando con $(P,x_0),(P,y_0)$ i due
  sistemi di divisioni che bisogna eseguire per trovare il M.C.D. (il
  quale è $1$) tra i numeri $P,x_0$; e rispettivamente $P,y_0$. Le
  relazioni~\eqref{alpha},~\eqref{beta}}
% 
{}

\begin{align*}
  &x_r\cdot y_{n-r} + x_{r+1}\cdot y_{n-r+1} = P\\
  &x_r\cdot y_s + (-1)^{n+r+s}\cdot x_{n-s+1}\cdot y_{n-r+1} \equiv 0 \mod P
\end{align*}

\colsni{diventano qui, considerando che $n$ è pari:}{}

\begin{align*}
  &x_r\cdot x_{n-r} + x_{r+1}\cdot x_{n-r+1} = P
  \tag{\(\alpha''\)}\label{alpha''}\\
  &x_r\cdot s_s + (-1)^{r+s}\cdot x_{n-r+1}\cdot x_{n-s+1} \equiv 0 \mod P.
    \tag{\(\beta''\)}\label{beta''}
\end{align*}

\cols{Dalla~\eqref{beta''} si ha in particolare per $r=s$}
%
{}

\begin{equation}
  x_r^2 + x_{n-r+1}^2 \equiv 0 \mod P.
  \tag{\(\gamma''\)}\label{gamma''}
\end{equation}

\cols{I resti intermedii sono $x_{\frac{n}{2}}, x_{\frac{n+2}{2}}$;
  ponendo nella~\eqref{alpha''} $r=\frac{n}{2}$, si ha:}
%
{}

\begin{equation}
  x_{\frac{n}{2}}^2 +  x_{\frac{n+2}{2}}^2 = P.
  \tag{\(\delta''\)}\label{delta''}
\end{equation}

\cols{Dunque i resti intermedii soddisfano all'equazione:}
%
{}

\[x^2 + y^2 = P.\]

\cols{Le relazioni $x_i=y_i$ insieme con
  le~\eqref{alpha''},~\eqref{beta''},~\eqref{gamma''},~\eqref{delta''},
  che ne abbiamo dedotte, si traducono nelle seguenti proprietà:

  \guillemotleft{} Dato un numero $P$, sia $x_0>\frac{P}{2}$ una
  radice della congruenza $x^2+1\equiv 0\mod P$, supposta
  solubile. Applicando ai numeri $P,x_0$ l'algoritmo della ricerca del
  M.C.D. si perviene al resto $1$ dopo un numero pari $n$ di
  divisioni.

  La successione dei quozientii e quella dei resti godono le seguentti
  proprietà:

  Il \ordinalnum{2} quoziente è uguale al penultimo resto diminuito di
  $1$.

  Fatta astrazione dai due primi quozienti, nella successione dei
  restanti i quozienti equidistanti dagli estremi sono uguali.

  La somma dei prodotti in croce di due resti consecutivi qualunque e
  dei corrispondenti equidistanti dagli estremi è uguale a $P$.

  La somma o differenza dei prodotti di due resti qualunque e dei
  corrispondenti equidistantii dagli estremi è multipla di $P$.

  Una coppia qualunque di resti equidistanti dagli estremi soddisfa
  alla congruenza  
}

\[x^2 + y^2 \equiv 0 \mod P.\]

\cols{La coppia dei resti intermedii soddisfa all'equazione:}
%
{}

\[x^2 + y^2 = P\]

\cols{Il metodo esposto fa dunque dipendere la conoscenza delle
  soluzioni di quest'ultima equazione dalla conoscenza delle radici
  della congruenza}
%
{}

\[x^2+1\equiv 0 \mod P;\]

\colsni{per ogni radice $x_0$ di questa congrenza si trova una
  soluzione dell'equazione $x^2+y^2=P$; a radici diverse corrispondono
  soluzioni diverse. Operando su tutte le radici $x_0>\frac{P}{2}$
  della congruenza $x^2+1\equiv 0\mod P$ si otterranno altrettante
  soluzioni distinte dell'equazione}
%
{}

\[x^2+y^2=P\]

\colsni{(vedremo più avanti che con questo metodo si ottengono
  \emph{tutte le soluzioni proprie}).

  \begin{corollary}
    Siccome per ogni numero primo $p$ della forma $4k+1$ è sempre
    solubile la congruenza $x^2+1\equiv 0 \mod P$, se ne deduce:

    Ogni numero primo della forma $4k+1$ è sempre scomponibile nella
    somma di due quadrati interi.\footnotemark
  \end{corollary}
}
%
{(we will see later that with this method we obtain \emph{all proper
    solutions}).

  \begin{corollary}
    Since for every prime number $p$ of the form $4k+1$ the congruence
    $x^2+1\equiv 0 \mod P$ always has a solution, we deduce:

    Every prime number of the form $4k+1$ is always decomposoable as a
    sum of two square integers.\footnotemark[\thefootnote]
  \end{corollary}
}
%
\footnotecols{Vedremo più innanzi come questa scomposizione sia
  unica.}{We will see later that this decomposition is unique.}

\secbreak

\cols{Ripigliamo le relazioni}

\begin{align}
  \tag{\ref{1}}
  x_{i-1} &= q_i x_i + x_{i+1}\\
  &&(i=0,1,2,\ldots, n)\notag\\
  \tag{\ref{2}}
  y_{n-i} &= q_i y_{n-i+1} + y_{n-i+2}.
\end{align}

\cols{Si abbia una congruenza della forma:}

\[\sum_{h=0}^n C_h x^{n-h}\cdot y^h \equiv 0 \mod P\]

\colsni{nella quale im primo membro è dunque una frazione (razionale
  intera) omogenea di grado $n$ nelle $x,y$ a coefficienti
  interi. Supponiamo che $x=x_1,y=1$, ne sia una soluzione, cioè che si
  abbia:}
%
{}

\[\sum_{h=0}^n C_h x_1^{n-h} \equiv 0 \mod P\]

\colsni{dico che sussisterà pure la congruenza}{}

\[\sum (-1)^{h(i-1)}\cdot C_h x_i^{n-h}\cdot y_{n-i+1}^h\equiv 0 \mod P.
  \qquad (i=1,2,\ldots,n)\]

\cols{Abbiamo dimostrato che}{}

\[x_i \equiv (-1)^{i-1}x_i\cdot y_{n-i+1} \mod P.\]

\cols{Avremo pertanto}

\begin{equation*}
  \sum (-1)^{h(i-1)}\cdot C_h x_i^{n-h} \cdot y_{n-i+1}^h \equiv
  \sum (-1)^{h(i-1)}\cdot C_h (-1)^{(n-h)(i-1)} \cdot x_1^{n-h} \cdot y_{n-i+1}^h  \equiv
  (-1)^{n(i-1)}\cdot y_{n-i+1}^n \cdot \sum_{h=0}^n C_h x_1^{n-h} \equiv 0 \mod P.
\end{equation*}

\cols{Concludiamo:

  Data una congruenza di grado $n$
  $\sum_{h=0}^nC_h x^{n-h}\equiv 0 \mod P$, ne sia $x_1<\frac{P}{2}$
  una radice. Posto $x+0=P-x_1$, sia $y_0>\frac{P}{2}$ la radice della
  congruenza $x_0y\equiv\pm 1\bmod P$ (che si calcola col procedimento
  sopra esposto): si applichi l'algoritmo della ricerca del
  M.C.D. alle due coppie $(P,x_0)$, $(P,y_0)$ (e basta far ciò per la
  prima coppia): sia $n$ il numero delle divisioni eseguite per
  giungere al resto $1$. Sia $x_1$ \footnotemark{} la serie dei resti
  del sistema $(P,x_0)$; $y_{n-i+1}$ quella del sistema $(P,y_0)$
  $(i=1,2,\dots,n)$.

  Allora
}
%
{}
\footnotetext{$x_i$, maybe?}

\[\sum_{h=0}^n (-1)^{h(i-1)} \cdot C_h x_i^{n-h} \cdot y_{n-i+1}^h \equiv 0 \mod P\]

\colsni{cioè}{}

\[\sum_{h=0}^n (-1)^{h(i-1)}\cdot C_h x_i^{n-h} \cdot y_{n-i+1}^h = k_i \cdot P.\]

\cols{Se uno dei $k_i$ fosse uguale ad $1$, avremmo così ottenuta una
  soluzione dell'equazione}
%
{}

\begin{equation}
  \tag{1}\label{binhomform}
  \sum_{h-0}^n C_h x^{n-h}\cdot y^h = P
\end{equation}

\colsni{il primo membro della quale è una forma binaria omogenea di grado $n$.

  Geometricamente la~\eqref{binhomform} è l'equazione di una curva
  algebrica di ordine $n$ che presenta le seguenti particolarità:

  Le tangenti nei punti all'$\infty$ hanno ciascuna con la curva un
  contatto $(\widetilde{n-1})$ punto e passano tutte per l'origine.

  Ogni retta per l'origine incontra la curva in un solo punto reale,
  se $n$ è dispari, in due soli punti reali se $n$ è pari.

  In quest'ultimo caso l'origine è centro di simmetria per la curva.

  Per vedere se col metodo delle divisioni successive sopra accennato
  si possa ottenere una soluzione della~\eqref{binhomform}, noi
  dovremo fare il cammino inverso, mostrando come da una soluzione di
  essa equazione possa dedursi una radice della congruenza
}
%
{}

\[\sum_{h=0}^n C_h x^{n-h} \equiv 0 \mod P.\]

\cols{\label{substituting-x0-y0}
  Sia dunque $(x_0,y_0)$ una soluzione propria
  della~\eqref{binhomform} (cioè con $x_0,y_0$ primi tra loro):
  supponiamo inoltre che $x_0$ sia primo con $C_n$ ed $y_0$ con $C_0$,
  altrimenti si potrebbe nella~\eqref{binhomform} raccogliere fuori di
  $\sum$ un fattore che dovrebbe dividere anche $P$.

  Di più nella~\eqref{binhomform} i coefficienti $C_h$ potranno essere
  in parte nulli, non però $C_0$ e $C_n$, altrimenti si potrebbe
  raccogliere fuori del segno di sommatoria una delle variabili $x,y$;
  il che noi vogliamo escludere.

  Ciò posto, distinguiamo due casi:

  I. $C_0$ e $C_n$ hanno lo stesso segno: possiamo allora supporre
  senz'altro $C_0$ e $C_n$ entrambi positivi: si divida
  $C_n y_0^{n-1}$ per $x_0$, e sia $q_0$ il quoziente, $x_1$ il resto,
  talché }
%
{}

\[C_n y_0^{n-1} - q_0 x_0 = x_1.\]

\cols{Si divida similmente $C_0 x_0^{n-1}$ per $y_0$, e sia $t_0$ il
  quoziente, $y_1$ il resto, di guisa che:}

\[C_0 x_0^{n-1} - t_0 y_0 = y_1.\]

\cols{Per le ipotesi poste sarà $x_0$ primo con $x_1$, $y_0$ con
  $y_1$. Applichiamo alle due coppie $(x_0,x_1)$, $(y_0,y_1)$
  l'algoritmo della ricerca del M.C.D., e siano}
%
{}

\begin{gather*}
  \left\{
    \begin{array}{l l l l l}
      x_2 & x_3 & \cdots & x_{r-1} & x_r=1\\
      q_2 & q_3 & \cdots & q_{r-1} & q_r
    \end{array}
  \right.\\
  \left\{
    \begin{array}{l l l l l}
      y_2 & y_3 & \cdots & y_{s-1} & y_s=1\\
      t_2 & t_3 & \cdots & t_{s-1} & t_s
    \end{array}
  \right.
\end{gather*}

\colsni{le successioni dei resti e dei quozienti nei sistemi di
  divisioni $(x_0,x_1)$, $(y_0,y_1)$ rispettivamente.

  Si avranno le relazioni}
%
{}

\begin{align}
  \tag{\ref{alpha}}
  x_i - q_{i+2} x_{i+1} &= x_{i+2} & (i=0,1,2,\ldots,r-1)\\
  \tag{\ref{beta}}
  y_j - t_{j+2} y_{j+1} &= y_{j+2} & (j=0,1,2,\ldots,s-1)
\end{align}

\colsni{con l'avvertenza di porre $x_{r+1}=1$\hfill
  $q_{r+1}=x_{r-1}-1$\hfill $y_{s+1}=1$\hfill $t_{s+1}=y_{s-1}-1$.

  Ciò posto, poniamo i due sistemi:}
%
{}

\begin{align}
  \tag{\ref{alpha'}}
  x_{s+1-\rho}' - t_{s+1-\rho}x_{s-\rho}' &= x_{s-\rho-1}'
  &(\rho=0,1,2,\ldots,s)\\
  \tag{\ref{beta'}}
  y_{r+1-\lambda}' - q_{r+1-\lambda}y_{r-\lambda}' &= y_{r-\lambda-1}'
  &(\lambda=0,1,2,\ldots,r)
\end{align}

\colsni{con l'avvertenza di fare $x_0'=x_0$\hfill $x_{-1}'=x_1$\hfill
  $y_0'=y_0$\hfill $y_{-1}'=y_1$}
%
{}

\[q_1 = t_1 = q_0 + t_0 + \sum_{h=1}^{n-1} C_h x_0^{n-h-1} \cdot y_0^{h-1}\]

\colsni{con ciò abbiamo due sistemi lineari, l'uno di $s+1$ equazioni
  ad $s+1$ incognite $x_1', x_2', \ldots, x_{s+1}'$; l'altro di $r+1$
  equazioni ad altrettante incognite $y_1',y_2',\ldots,y_{r+1}'$, le
  quali si possono determinare man mano per via ricorrente. Introdotto
  il simbolo di Gauss, si trova precisamente:}
%
{}

\begin{align}
  \tag{1}\label{contfrac1}
  x_{s+1-\rho}' &= [1, q_{r+1}, q_r, q_{r-1}, \ldots, q_2, t_1, t_2, \ldots, t_{s+1-\rho}]\\
  \tag{2}\label{contfrac2}
  y_{r+1-\lambda}' &= [1, t_{s+1}, t_s, t_{s-1}, \ldots, t_2, q_1, q_2, \ldots, q_{r+1-\lambda}]
\end{align}

\cols{mentre si ha}{}

\begin{align}
  \tag{3}\label{contfrac3}
  x_i &= [1, q_{r+1}, q_r, q_{r-1}, \ldots, q_{i+2}]
  &(i=0,1,2,\ldots,r-1)\\
  \tag{4}\label{contfrac4}
  y_j &= [1, t_{s+1}, t_s, t_{s-1}, \ldots, t_{j+2}].
  &(j=0,1,2,\ldots,s-1)
\end{align}

\cols{Possiamo riunire le $x$ accentate e non accentate, e similmente
  le $y$, attribuendo un unico index variabile da $m=r+s+1$ a zero;
  corrispondentemente si possono denotare con una sola lettera $q$
  affetta da indici, i quozienti $q_i$ e $t_j$; allora le
  relazioni~\eqref{contfrac1},~\eqref{contfrac2}~\eqref{contfrac3},~\eqref{contfrac4}
  possono raccogliersi nelle sole:}
% 
{}

\begin{align}
  \tag{\ref{contfrac1}}
  x_i = [1, q_m, q_{m-1}, q_{m-2}, \ldots, q_{i+1}]\\
  \notag && (i=m-1,m-2,\ldots,2,1,0)\\
  \tag{\ref{contfrac2}}
  y_{m-i-1} = [1, q_1, q_2, \ldots, q_{i+1}]
\end{align}

\colsni{dove adesso $q_m, q_{m-1}, q_{m-2}, \ldots, q_{m-r+1}$
  rappresentano quei quozienti prima denotati con
  $q_{r+1}, q_r, q_{r-1}, q_2$; mentre
  $q_{m-r}, q_{m-r-1}, \ldots, q_1$ rappresentano quei quozienti che
  prima si erano chiamati rispettivamente $t_1, t_2, \ldots,
  t_{s+1}$.  --- La soluzione iniziale $(x_0,y_0)$ dell'equazione
  $\sum_{h=0}^n C_h x^{n-h}\cdot y^h=P$ (supposta esistente) è adesso
  rappresentata dalla coppia $(x_{m-r}, y_{r+1})$.

  Ciò posto, formiamo lo specchio:}
%
{}

\[\begin{array}{r@{\;}l l l l l l}
    &x_0 & x_1 & x_2 & \cdots & x_m = 1 & x_{m+1} = 1\\
    1=&y_{m+1} & y_m=1 & y_{m-1} & \cdots & y_1 & y_0.
  \end{array}\]

\cols{Dico che due termini appartenenti alla stessa colonna soddisfano
  alla congruenza}
%
{}

\begin{align*}
  \sum_{h=0}^n {(-1)}^{hi} C_h x_i^{n-h}\cdot y_{m-i+1}^h \equiv\footnote{\(\equiv 0\)?} \mod P.
  &&(i=0,1,2,\ldots,m)
\end{align*}

\cols{Sia $H$ la radice della congruenza}{}

\[y_{r+1} \equiv x_{m-r}\cdot x \mod P\]

\colsni{sempre solubile, nell'ipotesi che $x_{m-r},y_{r+1}$ siano
  primi fra loro e con $P$. ---  Avremo:}

\begin{equation}
  \tag{\(a\)}\label{a}
  y_{r+1} \equiv H x_{m-r} \mod P.
\end{equation}

\cols{Ora dico che sussiste la congruenza}{}

\begin{align}
  \tag{\(b\)}\label{b}
  y_{i+1} - {(-1)}^{r-i}\cdot H x_{m-i} \equiv 0 \mod P.
  &&(i=0,1,2,\ldots,m)
\end{align}

\cols{Intanto, come mostra la~\eqref{a}, la~\eqref{b} è vera per
  $i=r$. Facciamo ora vedere che la stessa congruenza sussiste anche
  per $i=r-1$ e per $i=r+1$. --Ricordiamo che $x_m,y_{r+1}$ sono quei
  numeri che prima avevamo chiamato $x_0,y_0$; e che
  $x_{m-r-1},y_{r+2},x_{m-r+1},y_r$ rappresentano quelle quantità che
  prima si erano denotate con $x_1', y_1,x_1,y_1'$
  rispettivamente.  --- Talché per ipotesi e per costruzione avremo:}
%
{}

\begin{align*}
  x_{m-r+1} &= C_n y_{r+1}^{n-1} - q_0 x_{m-r}
              \qquad x_{r+1} = C_0 x_{m-r}^{n-1} - t_0 y_{r+1}\\
  x_{m-r-1} &= \left[ q_0 t_0 + \sum_{h=1}^{n-1} C_h x_{m-r}^{n-h-1} \cdot y_{r-1}^{h-1} \right]
              \times x_{m-r} + C_n y_{r+1}^{n-1} - q_0 x_{m-r} =
              t_0 x_{m-r} + \sum_{h=0}^{n-1} C_h x_{m-r}^{n-h} \cdot y_{r+1}^{h-1}
\end{align*}
%
\begin{equation*}
  y_r = \left[ q_0 + t_0 + \sum_{h=1}^{n-1} C_h x_{m-r}^{n-h-1} \cdot y_{r+1}^{h-1}\right]
  \times y_{r+1} + C_0 x_{m-r}^{n-1} - t_0 y_{r+1} =
  q_0 y_{r+1} + \sum_{h=0}^{n-1} C_h x_{m-r}^{n-h-1}\cdot y_{r+1}^h
\end{equation*}
%
\begin{equation*}
  \sum_{h=0}^n C_h x_{m-r}^{n-h}\cdot y_{r+1}^h = P.
\end{equation*}

\cols{Avremo perciò}{}

\begin{equation*}
  y_r + H x_{m-r+1} = q_0 y_{r+1} + \frac{P - C_n y_{r+1}^n}{x_{m-r}}
  + H (C_n y_{r+1}^{n-1} - q_0 y_{m-r})
\end{equation*}
%
\begin{multline*}
  (y_r + H x_{m-r+1}) x_{m-r} = q_0 y_{r+1} x_{m-r} + P - C_n y_{r+1}^n
  + H x_{m-r} (C_n y_{r+1}^{n-1} - q_0 x_{m-r}) \equiv\\
  \equiv q_0 x_{m-r} (y_{r+1} - H x_{m-r}) - C_n y_{r+1}^{n-1}(y_{r+1} - H x_{m-r}) \equiv
  (q_0 x_{m-r} - C_n y_{r+1}^{n-1}) \times (y_{r+1} - H x_{m-r}) \equiv 0 \mod P
\end{multline*}

\colsni{ed essendo $x_{m-r}$ primo con $P$, se ne deduce}
%
{}

\[y_r + H x_{m-r+1} \equiv 0 \mod P.\]

\cols{Similmente}{Similarly}

\begin{equation*}
  y_{r+2} + H x_{m-r-1} = t_0 (y_{r+1} - H x_{m-r}) + C_0 x_{m-r}^{n-1}
  + H \frac{P - C_0 x_{m-r}^n}{y_{r+1}} \equiv
  C_0 x_{m-r}^{n-1} + H\frac{P - C_0 x_{m-r}^n}{y_{r+1}} \mod P
\end{equation*}
%
\begin{equation*}
  (y_{r+2} + H x_{m-r-1}) y_{r+1} \equiv C_0 x_{m-r}^{n-1} y_{r+1} - C_0 x_{m-r}^n \cdot H
  \equiv C_0 x_{m-r}^{n-1}(y_{r+1} - H x_{m-r}) \equiv 0 \mod P
\end{equation*}

\colsni{ed essendo $y_{r+1}$ primo con $P$}{}

\[y_{r+2} + H x_{m-r-1} \equiv 0 \mod P.\]

\cols{Dunque intanto la congruenza}{}

\begin{equation}
  \tag{\ref{b}}
  y_{i+1} - {(-1)}^{r-1} \cdot H \cdot x_{m-i} \equiv 0 \mod P.
\end{equation}

\colsni{è vera per $i=r,r-1,r+1$.

  Ora supponiamo che la~\eqref{b} sia vera fino all'indice
  $r\pm \lambda$; si vuol provare che essa vale anche per l'indice
  $r\pm (\lambda+1)$.  --- Si ha}
%
{}

\begin{align*}
  x_{m-(r+\lambda+1)} &= q_{m-(r+\lambda+1)} \times x_{m-(r+\lambda)} + x_{m-(r+\lambda-1)}\\
  y_{r+1+(\lambda+1)} &= y_{r+1+(\lambda-1)} - q_{m-(r+\lambda)} \cdot y_{r+\lambda+1}\\
  x_{m-(r-(\lambda+1))} &= x_{m-(r-(\lambda-1))} - q_{m-(r-\lambda)} \cdot x_{m-(r-\lambda)}\\
  y_{r+1-(\lambda+1)} &= q_{m-(r-\lambda)} \cdot y_{r+1-\lambda} + y_{r+1-(\lambda-1)}.
\end{align*}

\cols{Da cui si ricava facilmente, mediante sostituzione}
%
{}

\begin{gather*}
  y_{r+1+(\lambda+1)} - {(-1)}^{r-(r+\lambda+1)}\cdot H \cdot x_{m-(r+\lambda+1)} =\\[1em]
  = \overline{y_{r+1+(\lambda-1)} -{(-1)}^{r-(r+\lambda-1)} \cdot H \cdot x_{m-(r+\lambda-1)}}
    + q_{m-(r+\lambda)} \overline{(y_{r+1+\lambda} -{(-1)}^{r-(r+\lambda)} \cdot H x_{m-(r+\lambda)})}\\[1.5em]
  y_{r+1-(\lambda+1)} - {(-1)}^{r-(r-(\lambda+1))}\cdot H x_{m-(r-(\lambda+1))} =\\[1em]
  = \overline{y_{r+1-(\lambda-1)} -{(-1)}^{r-(r-(\lambda-1))} \cdot H x_{m-(r-(\lambda-1))}}
    + q_{m-(r-\lambda)} \overline{(y_{r+1-\lambda} -{(-1)}^{r-(r-\lambda)} \cdot H x_{m-(r-\lambda)})}.
\end{gather*}

\cols{Ed essendo per ipotesi i binomii sopra lineati $\equiv 0 \mod P$, si conclude}
%
{}

\[y_{r+1+(\lambda+1)} - {(-1)}^{(\lambda+1)} \cdot H x_{m-(r+\lambda+1)} \equiv
  y_{r+1-(\lambda+1)} - {(-1)}^{+(\lambda+1)} \cdot H x_{m-(r-(\lambda+1))} \equiv
  0 \mod P.\]

\cols{Perciò se si ammette che la congruenza}{}

\[y_{r+1+\underline{\lambda}} - {(-1)}^\lambda H \cdot x_{m-(r+\underline{\lambda})} \equiv 0 \mod P\]

\colsni{sia vera per $\lambda = 0,1,2,\ldots,\bar{\lambda}$, essa è
  vera anche per $\lambda=\bar{\lambda}+1$.
  
  Avendo provato ch'essa vale per $\lambda=0,1$, si può affermare
  ch'essa sussiste per ogni $\lambda$; e quindi anche per la
  congruenza}
%
{}

\[y_{i+1} - (-1)^{r-i} \cdot H x_{m-i} \equiv 0 \mod P\]

\colsni{vale per $i=0,1,2,\dots,m$.

Avremo pertanto:}
{}

\begin{equation*}
  \sum_{h=0}^n (-1)^{h(r-i)} \cdot C_h x_{m-i}^{n-h} \cdot y_{i+1}^h
  \equiv \sum_{h=0}^n (-1)^{h(r-i)} \cdot C_h x_{m-i}^{n-h} (-1)^{h(r-i)} \cdot H^h \cdot x_{m-i}^h
  \equiv x_{m-i}^n \sum_{h=0}^n C_h H^h \mod P.
\end{equation*}

\cols{Ma per ipotesi:}{}

\begin{equation*}
  \sum_{h=0}^n C_h x_{m-r}^{n-h} \cdot y_{r+1}^h = P \equiv 0 \mod P
  \qquad y_{r+1} \equiv H \cdot x_{m-r} \mod P
\end{equation*}

\colsni{quindi:}{}

\begin{equation*}
  \sum C_h x_{m-r}^{n-h} \cdot y_{r+1}^h \equiv
  \sum C_h x_{m-r}^{n-h} \cdot H^h \cdot x_{m-r}^h \equiv
  x_{m-r}^n \cdot \sum C_h H^h \equiv 
  0 \mod P
\end{equation*}

\colsni{ed essendo $x_{m-r}^n$ primo con $P$}{}

\[\sum C_h H^h \equiv 0 \mod P.\]

\cols{Laonde anche}{}

\begin{align*}
  \sum_{h=0}^n (-1)^{h(r-i)} \cdot C_h x_{m-i}^{n-h} \cdot y_{i+1}^
  \equiv 0 \mod P. 
  &&(i=0,1,2,\dots,m)
\end{align*}

\cols{In particolare per $i=0$ e per $i=m-1$ si hanno le due congruenze:}{}

\begin{align*}
  &\sum_{h=0}^n (-1)^{hr} C_h y_1^h \equiv 0 \mod P\\
  &\sum_{h=0}^n (-1)^{h(r-m+1)} \cdot C_h x_1^{n-h} \equiv 0 \mod P
\end{align*}

\colsni{cioè $(x_1),(y_1)$ sono rispettivamente radici delle congruenze}{}

\[\sum_{h=0}^n C_h x^{n-h} \equiv 0 \qquad \sum_{h=0}^n C_h y^h \equiv 0 \mod P.\]

\cols{Per un teorema precedente si ha poi}{}

\[x_1 y_1 \equiv \pm 1 \mod P\]

\colsni{e inoltre

  \begin{align*}
    x_{m-i} \cdot y_i + x_{m-i+1} \cdot y_{i+1} = \text{constante.}\\
    (\text{per } i=0,1,2,\dots,m)
  \end{align*}

  Ma per $i=r,r+1$ si ha
}{}

\begin{equation*}
  x_{m-r} \cdot y_r + x_{m-r+1} \cdot y_{r+1} =
  y_{r+1} \Bigl( C_n y_{r+1}^{n-1} - q_0 x_{m-r} \Bigr) +
  x_{m-r} \cdot \Bigl(q_0 y_{r+1} + \sum_{h=0}^{n-1} C_h x_{m-r}^{n-h-1} \cdot y_{r+1}^h \Bigr) =
  \sum_{h=0}^n C_h x_{m-r}^{n-h} \cdot y_{r+1} = P
\end{equation*}

\begin{equation*}
  y_{m-r-1}\cdot y_{r+1} + x_{m-r} \cdot y_{r+2} =
  y_{r+1} \Bigl(t_0 x_{m-r} + \sum_{h=1}^n C_h x_{m-r}^{n-h} \cdot y_{r+1}^{h-1} \Bigr) + x_{m-r} \Bigl( C_0 x_{m-r}^{n-1} - t_0 y_{r+1} \Bigr) =
  \sum_{h=0}^n C_h x_{m-r}^{n-h} \cdot y_{r+1}^h = P.
\end{equation*}

\cols{Perciò, per ogni $i$ è}{}

\[x_{m-i} \cdot y_i + x_{m-i+1} \cdot y_{i+1} = P.\]

\cols{In particolare per $i=0,m$ si ha}{}

\[x_0 + x_1 = y_0 + y_1 = P.\]

\cols{ --- Riprendendo le relazioni:}{}

\begin{align}
  \tag{\ref{alpha}}
  \left\{\begin{array}{l}
           x_{i-1} - q_i x_i = x_{i+1} \\
           x_{m-i} - q_i y_{m-i+1} = y_{m-i+2} 
         \end{array}\right.
  &&(i=1,2,\dots,m)
\end{align}

\colsni{si scorge come le quantità $q_i$, eccetto la $q_{m-r}$,
  rappresentano rispettivamente le due serie dei quozienti nei due
  sistemi di divisioni $(x_{m-r},x_{m-r+1}),(y_{r+1},y_{r+2})$, sono
  tutte positive. Che se anche $q_{m-r}$ fosse positiva, allora le
  relazioni \eqref{alpha} insieme con le $x_0+x_1=P$, $y_0+y_1=P$ ci
  dicono come i numeri $x_i$ non sono che i resti che si ottengono
  applicando alla coppia $(P,x_0)$ l'algoritmo della ricerca del
  M.C.D.; o come diremo più brevemente, è la successione dei resti nel
  sistema di divisioni $(P,x_0)$. --- Similmente i numeri $y_{m-i}$ sono
  i resti del sistema di divisioni $(P,y_0)$. --- Inoltre dobbiamo
  ricordare che $(x_0)$ è una radice della congruenza
  $\sum_{h=0}^n C_h x^{n-h}\equiv 0 \mod P$, e che
  $x_0 y_0 \equiv \pm 1 \mod P$.

  Prima di trarre le conseguenze delle proprietà stabilite, cerchiamo
  quinidi la condizione affinchè risulti $q_{m-r}>0$. Poniamo per
  semplicità $x_{m-r}=a\quad y_{r+1}=b$: si ha}
% 
{}

\begin{multline*}
  q_{m-r} = q_0 + t_0 + \sum_{h=1}^{n-1} C_h a^{n-1-h} \cdot b^{h-1} 
  = q_0 + t_0 + \frac{P - C_0 a^n - C_n b^n}{ab} 
  = \frac{1}{ab} (q_0 ab + t_0 ab + P - C_0 a^n - C_n b^n) =\\
  = \frac{P - a(C_0 a^{n-1} - t_0 b) - b(C_n b^{n-1} - q_0 a)}{ab}
  = \frac{P - (a y_{r+2} + b x_{m-r+1})}{a \cdot b}.
\end{multline*}

\cols{Perchè dunque risulti $q_{m-r}>0$ bisognerà che sia}{}

\[P > a y_{r+2} + b x_{m-r+1}.\]

\cols{Siccome poi è $x_{m-r-1} < a$, $y_{r+2} < b$, e quindi}{}

\[a y_{r+2} + b x_{m-r+1} < 2ab\]

\colsni{se fosse}{}

\[P > 2ab\]

\colsni{a maggior ragione risulterebbe}{}

\[P > a y_{r+2} + b x_{m-r+1},\]

\colsni{e quindi $q_{m-r} > 0$,}{}

\cols{Osservando poi che quando $q_{m-r} > 0$ è
  $y_r = q_{m-r} b + y_{r+2} > b$, dalla relazione}
% 
{}

\[y_r \cdot a + x_{m-r+1} \cdot b = P\]

\colsni{si trae $a y_r < P$ e a forziori $ab < P$.

  Affinchè dunque risulti $q_{m-r} > 0$ è necessario che sia $ab<P$, e
  sufficiente che si abbia $2ab<P$.

  Dopo ciò, richiamando i risultati precedentemente ottenuti possiamo
  concludere:

  \guillemotleft{} Sia data l'equazione
  $\sum_{h=0}^n C_h x^{n-h}\cdot y^h = P$, il cui primo membro è una
  forma omogenea di grado $n$ nelle indeterminate $x$ e $y$, con
  coefficienti interi, il primo e l'ultimo positivi (non
  nulli). Consideriamo una radice $x_0 > \frac{P}{2}$ della congruenza
  $\sum_{h=0}^n C_h x^{n-h} \equiv 0 \mod P$ e la sua coniugata
  $y_0 > \frac{P}{2}$, tale cioè che $x_0 y_0 \equiv \pm 1 \mod P$, la
  quale soddisferà perciò alla congruenza
  $\sum_{h=0}^n C_h y^h \equiv 0 \mod P$.

  Applichiamo alle coppie $(P,x_0)$, $(P,y_0)$ l'algoritmo della
  ricerca del M.C.D.; si perverrà in entrambi i casi al resto $1$,
  dopo uno stesso numero $m$ di divisioni.  --- Siano
  $x_1,x_2,\ldots,x_m=1$; $y_1,y_2,\ldots,y_m=1$ i resti nel sistema
  di divisioni $(P,x_0)$ e rispett. $(P,y_0)$. ---  Formando lo specchio
}
% 
{}

\begin{equation}
  \tag{S}\label{S}
  \left\{
    \begin{array}{rlllll}
      x_1 & x_2 & x_3 & \cdots & x_{m-1} & x_m = 1  \\
      1 = y_m & y_{m-1} & y_{m-2} & \cdots & y_2 & y_1 
    \end{array}
  \right.
\end{equation}

\colsni{due termini qualunque appartenenti alla stessa colonna,
  soddisfano alla congruenza}
% 
{}

\[\sum (-1)^{h \lambda} \cdot C_h x_\lambda^{n-h} \cdot y_{m-\lambda+1}^h \equiv 0 \mod P.\]

\cols{Inoltre, se l'equazione}{}

\[\sum C_h x^{n-h} \cdot y^h = P\]

\colsni{ammette una soluzione propria $(a,b)$ tale che $2ab<P$, essa
  soluzione si troverà certamente in una delle colonne dello
  specchio~\eqref{S}, o di un altro analogo ottenuto da un'altra
  radice della congruenza}
% 
{}

\[\sum C_h x^{n-h} \equiv 0 \mod P.\]

\cols{Adunque, applicando a tutte le radici di questa congruenza lo
  stesso processo applicato alla radice $x_0$, si troveranno
  certamente, tra le corrispondenti successioni dei resi, tutte le
  soluzioni, se esistono, dell'equaz.
  $\sum C_h x^{n-h} \cdot y^h\equiv P$, per le quali risulta
  $2(xy)<P$.  --- Infatti, se con questo metodo non si ottenesse una
  certa soluzione $(a,b)$ soddisfacente alla condizione $2ab<P$,
  partendo da questa soluzione e operando come dianzi, si arriverebbe
  a trovare una radice $x_0$ della cong.
  $\sum C_h x^{n-h} \equiv 0 \mod P$ con $x_0>\frac{P}{2}$; e allora,
  se $y_0>\frac{P}{2}$ è la coniugata di $x_0$ (tale cioè che
  $x_0 y_0 \equiv \pm 1 \mod P$) applicando l'algoritmo della ricerca
  del M.C.D. alle coppie $(P,x_0)$, $(P,y_0)$ ritroveremo
  necessariamente tra i resti dei due sistemi di divisioni la
  soluzione $(a,b)$ che per falsa ipotesi abbiamo supposto non
  ottenuta.

  Il metodo esposto delle divisioni successive dà dunque tutte le
  soluzioni proprie dell'eq. $\sum_{h=0}^n C_h x^{n-h} \cdot y^h = P$
  soddisfacenti alla condizione $2|xy|<P$, note che siano le radici
  della congruenza $\sum_{h=0}^n C_h x^{n-h} \equiv 0 \mod P$.

  Se $x_0>\frac{P}{2}$ è una di tali radici, ed $y_0>\frac{P}{2}$ il
  numero $<P$ tale che $x_0 y_0 \equiv \pm 1 \mod P$, applicando alle
  coppie $(P,x_0)$, $(P,y_0)$ l'algoritmo della ricerca del
  M.C.D.\footnotemark, e formando coi resti lo specchio
  precedente~\eqref{S}, due termini qualunque
  $x_\lambda, y_{m-\lambda+1}$ appartenenti alla stessa colonna
  soddisfano alla congruenza}
% 
{}

\footnotecols{Notiamo una volta per sempre che basta eseguire il
  sistema di divisioni $(P,x_0)$ poichè l'altro risulta determinato
  dallo stesso processo che conduce a trovare il numero $y_0$; e
  precisamente, se $q_0=1,q_1,q_2,\ldots,q_{m-1}$ sono i quozienti del
  sistema $(P,x_0)$, e si pone $q_m=x_{m-1}-1$, si ha:

  \[y_i = [1,q_1,q_2,\ldots,q_{m-i}]\qquad;\qquad 
    y_0 = [1,q_1,q_2,\ldots,q_{m-1},q_m].\]}
% 
{}

\[\sum_{h=0}^n (-1)^{h\lambda} C_h x_\lambda^{n-h} \cdot y_{m-\lambda+1}^h \equiv 0 \mod P\]

\colsni{per cui si può scrivere:}{}

\[\sum (-1)^{h\lambda} C_h \cdot x_\lambda^{n-h} \cdot y_{m-\lambda+1}^h = k_\lambda \cdot P\]

\colsni{con $k_\lambda$ intero (positivo o negativo).

  Se alla radice considerata $x_0$ corrisponde una soluzione propria
  $(a,b)$ dell'equazione}
% 
{}

\begin{equation}
  \tag{A}\label{A}
  \sum_{h=0}^n C_h x^{n-h} \cdot y^h = P
\end{equation}

\colsni{soddisfacente alla condizione $2ab<P$, essa, come abbiam
  visto, deve necessariamente trovarsi in una delle colonne dello
  specchio precedente~\eqref{S}; cioè uno dei $k_\lambda$ dev'essere
  uguale ad $1$.  --- Per accertare dunque l'esistenza di questa
  soluzione, bisognerebbe calcolare, per ogni coppia
  $(x_\lambda, y_{m-\lambda+1})$ dello specchio il corrispondente
  $k_\lambda$. Ma questa verifica non è in generale semplice, se il
  numero $m$ delle divisioni $(P,x_0)$ è grande, e se il grado e il
  numero dei termini della forma $\sum C_h x^{n-h}\cdot y^h$ è
  abbastanza elevato. ---  Di qui la necessità di dare dei criterii che
  permettano di semplificare questa verifica.

  Osserviamo anzitutto che data una radice $x_u$ della congruenza}
% 
{}

\[\sum (-1)^{h\lambda} C_h x^{n-h} \equiv 0 \mod P,\]

\colsni{si possono trovare tutte le soluzioni della
  $\sum C_h x^{n-h} \cdot y^h \equiv 0 \mod P$ \emph{appartenenti}
  alla radice $x_u$.

  Preso in fatti un $\bar{y}$ qualunque inferiore a $P$, e ponendo
  $x_u \bar{y} = \bar{x} \mod P$, si trova facilmente}
% 
{}

\[\sum C_h \bar{x}^{n-h} \cdot \bar{y}^h \equiv 0 \mod P.\]

\cols{Si tratta ora di vedere quali siano quelle di queste soluzioni
  che si ottengono col metodo delle divisioni successive.  --- Anche qui
  partiremo da una di tali soluzioni, e mostreremo come se ne possa
  dedurre una radice della congruenza}
% 
{}

\[\sum C_h x^{n-h} \equiv 0 \mod P.\]

\cols{Sia dunque $(x_0,y_0)$ una soluzione della congruenza
  $\sum C_h x^{n-h} \cdot y^h \equiv 0 \mod P$ cioè sia}
% 
{}

\[\sum C_h x_0^{n-h}\cdot y_0^h = k_0 P\]

\colsni{($k_0$ \text{intero}).

  Sulla forma omogenea del \ordinalnum{1} membro manterremo le stesse
  ipotesi già fatte, cioè supponiamo $C_0$ e $C_n$ positivi e non
  nulli; $x_0$ primo con $C_n y_0$; $y_0$ primo con $C_0 x_0$; qui
  aggiungeremo la condizione che sia $k_0$ primo con $x_0 y_0$. Si
  divida $C_n y_0^{n-i}$ per $x_0$, e sia $q_0$ il quoziente,
  $x_\lambda$ il resto, talchè}
% 
{}

\[C_n y_0^{n-1} - q_0 x_0 = x_\lambda.\]

\cols{Così sia $t_0$ il quoziente, $y_\lambda$ il resto della
  divisione di $C_0 x_0^{n-1}$ per $y_0$, talchè}
%
{}

\[C_n x_0^{n-1} - t_0 y_0 = y_\lambda.\]

\cols{Siano $x_1<x_0$, $y_1<y_0$ le radici delle due congruenze}
%
{}

\begin{align*}
  x_\lambda &\equiv k_0 x \mod x_0\\
  y_\lambda &\equiv k_0 y \mod y_0\\
\end{align*}

\colsni{le quali, per le ipotesi poste, sono sempre (e in modo unico)
  solubili. Allora}
%
{}

\[k_0 x_1 - x_\lambda = m_0 x_0\qquad
  k_0 y_1 - y_\lambda = n_0 y_0.\]

\cols{Vogilamo dimostrare che l'espressione}{}

\[Q = q_0 + t_0 + \sum_{h=1}^{n-1} C_h x_0^{n-h-1} \cdot y_0^{h-1} - m_0 - n_0\]

\colsni{è divisibile per $k_0$.  --- Infatti si ha:}{}

\begin{multline*}
  Q x_0 y_0 = q_0 x_0 y_0 + t_0 x_0 y_0 + k_0 P - C_0 x_0^n - C_n y_0^n - (m_0 + n_0) x_0 y_0
  = k_0 P - x_0 y_\lambda - x_\lambda y_0 - (m_0 + n_0) x_0 y_0 =\\
  = k_0 P - k_0 x_1 x_0 - k_0 x_1 y_0
  = k_0 (P - x_1 y_0 - y_1 x_0)
  \equiv 0 \mod k_0.
\end{multline*}

\cols{Ma per ipotesi è $k_0$ primo con $x_0 y_0$; perciò $Q \equiv 0 \mod k_0$.

  Adunque l'espressione}
%
{}

\[q_1 = \frac{Q}{k_0} = \frac{P - x_1 y_0 - y_1 x_0}{x_0 y_0}\]

\colsni{è un numero intero. --- Ciò posto, si sviluppino i due sistemi
  di divisioni $(x_0,x_1)$, $(y_0,y_1)$; siano $x_2,x_3,\ldots,x_r=1$
  $y_2,y_3,\ldots,y_s=1$ le due serie dei resti; $q_2,q_3,\ldots,q_r$
  $t_2,t_3,\ldots,t_s$ le due serie dei quozienti.\footnotemark{} ---
  Procedendo ora come pel caso già trattato (vedi
  pag.~\pageref{substituting-x0-y0} e seg.)  in cui $(x_0,y_0)$ era
  una soluzione della $\sum C_h x^{n-h}\cdot y^h=P$, cioè $k_0=1$, si
  trova che sarà $q_1>0$ se $x_0y_1+x_1y_0<P$; il che avverrà
  certamente se $2|x_0y_0|<P$. Come allora, concluderemo che col
  metodo delle divisioni successive si ottengono tutte le soluzioni
  proprie della congruenza $\sum C_h x^{n-h}\cdot y^h\equiv 0\mod P$,
  relative a una radice $x_0$ della $\sum C_h x^{n-h}\equiv 0 \mod P$,
  che soddisfano alla condizione $2|xy|<P$. --- Se al solito, il
  sistema di divisioni $(P,x_0)$ dà i resti $x_1,x_2,\ldots,x_m=1$, e
  i quozienti $1,q_1,q_2,\ldots,q_{m-1}$, ponendo $q_m=x_{m-1}$,
  formiamo i numeri $y_m=1$, $y_{m-1}=[1,q_1]$, $y_{m-2}=[q,q_1,q_2]$,
  \dots, $y_1=[1,q_1,q_2,\ldots,q_{m-1}]$,
  $y_0=[1,q_1,q_2,\ldots,q_m]$. Allora nel solito specchio}
%
{}

\footnotetext{Only $x_2,\ldots$ are separated by commas in the
  original text.}

\begin{equation}
  \tag{\ref{S}}
  \left\{
    \begin{array}{rllll}
      x_1 & x_2 & \cdots & x_{m-1} & x_m = 1\\
      1 = y_m & y_{m-1} & \cdots & y_2 & y_1
    \end{array}
  \right.
\end{equation}

\colsni{due termini qualunque $x_\lambda,y_{m-\lambda+1}$ nella stessa
  colonna soddisfano alla relazione}
%
{}

\[\sum_{h=0}^n (-1)^{h\lambda} C_h x_\lambda^{n-h}\cdot y_{m-\lambda+1}^h
  = k_\lambda \cdot P\]

\colsni{con $k_\lambda$ intero.

  Ora in base ai risultati precedenti possiamo dire qualche cosa di $k_\lambda$.

  Infatti, dovrà essere:}
%
{}

\begin{align*}
  C_0 x_\lambda^{n-1} \equiv k_\lambda \cdot y_{m-\lambda+2} &\mod y_{m-\lambda+1}\\
  C_n y_{m-\lambda+1}^{n-1} \equiv k_\lambda \cdot x_{\lambda+1} &\mod x_\lambda
\end{align*}

\colsni{e reciprocamente, se $(x_\lambda,y_{m-\lambda+1})$ sono due
  termini della stessa colonna nello specchio~\eqref{S}, si può
  trovare, a meno di un multiplo di $x_\lambda y_{m-\lambda+1}$, il
  corrispondente $k_\lambda$. Prendendo infatti i due termini
  $x_{\lambda+1},y_{m-\lambda+2}$, l'uno seguente $x_\lambda$ nella
  \ordinalnum{1}[f] linea di~\eqref{S}, l'altro precedente
  $y_{m-\lambda+1}$ nella \ordinalnum{2}[f], si pongano le
  congruenze:}
%
{}

\begin{align*}
  C_0 x_\lambda^{n-1} \equiv X \cdot y_{m-\lambda+2} &\mod y_{m-\lambda+1}\\
  C_n y_{m-\lambda+1}^{n-1} \equiv X \cdot x_{\lambda+1} &\mod x_\lambda.
\end{align*}

\cols{Se $a_\lambda$ è il minimo numero che soddisfa
  contemporeneamente queste due congruenze (numero che sempre esiste),
  avremo $k_\lambda=a_\lambda + k x_\lambda y_{m-\lambda+1}$}
%
{}

\[\sum (-1)^{h\lambda} C_h x_\lambda^{n-h} \cdot y_{m-\lambda+1}^h
  = (a_\lambda + k x_\lambda \cdot y_{m-\lambda+1}) \cdot P\]

\colsni{dove $k$ è un intero (positivo o negativo).

  In particolare, se alla radice considerata $x_0$ corrisponde une
  soluzione propria $(x_u,y_{m-u+1})$ dell'equazione
  $\sum C_h x^{n-h} \cdot y^h=\pm P$, soddisfacente alla condizione
  $2|xy|<P$, essa dovrà trovarsi in una colonna dello specchio
  precedente~\eqref{S}, ed essere:
}
%
{}

\begin{align*}
  C_0 x_u^{n-1} \equiv \pm y_{m-u+2} &\mod y_{m-u+1}\\
  C_n y_{m-u+1}^{n-1} \equiv \pm x_{u+1} &\mod x_u.
\end{align*}

\cols{Formate dunque, corrispondentemente ad ogni colonna dello
  specchio~\eqref{S} le quantità}
%
{}

\[D_u = C_0 x_u^{n-1} \pm y_{m-n+2}\qquad
  D_u' = C_n y_{m-u+1}^{n-1} \pm x_{u+1},\]

\colsni{la soluzione dell'equazione
  $\sum C_h x^{n-h} \cdot y^h = \pm P$,\footnotemark{} se esiste, si
  troverà in una di quelle colonne, per cui risulta $D_u$ divisibile
  per $y_{m-u+1}$ e $D_u'$ divisibile per $x_u$.

  --- Ora diamo un altro criterio che dà il posto preciso occupato
  nello specchio~\eqref{S} dalla soluzione, se esiste, dell'equazione
  $\sum C_h x^{n-h}\cdot y^h = P$, imponendo però a questa soluzione
  un'ulteriore condizione.

  Ritornando allo specchio~\eqref{S}, è evidente che i numeri
  $x_1,x_2,\ldots,x_m$ della prima linea vanno decrescendo, quelli
  $y_m,y_{m-1},\ldots,y_2,y_1$ della \ordinalnum{2}[f] linea vanno
  crescendo. --- Siccome nella \ordinalnum{2}[f] linea vi è almeno
  l'elemento $y_1$ maggiore del corrispondente $x_m$ della
  \ordinalnum{1}[f], è chiaro che nella \ordinalnum{2}[f] linea vi
  sarà un primo elemento che supera il corrispondente della
  \ordinalnum{1}[f].

  Sia esso $y_{m-u+1}$: allora tutti gli elementi che precedono
  (seguono) $y_{m-u+1}$ nella \ordinalnum{2}[f] linea sono minori
  (maggiori) dei corrispondenti della \ordinalnum{1}[f]. Si avrà
  così:}
%
{}

\footnotetext{Illegible: guessing $=$ and $\pm$.}

\[x_{u-1} > y_{m-u+2}\qquad x_u < y_{m-u+1}.\]

\cols{Ora è:}{}

\[x_{u-1} = q x_u + x_{u+1}\qquad
  q = \frac{P - x_{u+1} \cdot y_{m-u+1} - x_u \cdot y_{m-u+2}}{x_u y_{m-u+1}}\]

\colsni{cioè}{}

\[x_{u-1}
  = \frac{P - x_{u+1} y_{m-u+1} - x_u y_{m-u+2} + x_{u+1} y_{m-u+1}}{y_{m-u+1}}
  = \frac{P - x_u y_{m-u+2}}{y_{m-u+1}}.\]

\cols{Sarà dunque}{}

\[\frac{P - x_u y_{m-u+2}}{y_{m-u+1}} > y_{m-u+2}\qquad
  P > y_{m-u+2} (x_u + y_{m-u+1})\]

\colsni{e similmente, essendo}

\[y_{m-u} > x_{u+1}\qquad
  y_{m-u} = \frac{P - x_{u+1}\cdot y_{m-u+1}}{x_u}\]

\colsni{sarà:}

\[P - (x_{u+1}\cdot y_{m-u+1}) > x_u x_{u+1}\qquad
  P > x_{u+1} (x_u + y_{m-u+1}).\]

\cols{Reciprocamente, se per due termini $x_u,y_{m-u+1}$
  corrispondenti dello specchio~\eqref{S} si ha}
%
{}

\[x_{u+1} (x_u + y_{m-u+1}) < P\qquad
  y_{m-u+2} ( x_u + y_{m-u+1}) < P,\]

\colsni{cioè se $P$ è maggiore della più grande delle due quantità}
%
{}

\[x_{u+1} (x_u + y_{m-u+1}),
  y_{m-u+2} ( x_u + y_{m-u+1}),\]

\colsni{si dimostra che}{}

\[x_{u-1} > y_{m-u+2}\qquad y_{m-u} > x_{u+1};\]

\colsni{e quindi che $y_{m-u+1}$ è il primo termine della
  \ordinalnum{2}[f] linea dello specchio~\eqref{S}, procedendo da
  sinistra a destra, che supera il corrispondente della
  \ordinalnum{1}[f]; se $y_{m-u+1}>x_u$; e che $x_u$ è il primo numero
  della \ordinalnum{1}[f] linea, procedendo da destra a sinistra, che
  supera il corrispondente della \ordinalnum{2}[f], se
  $y_{m-u+1}<x_u$.

  Infatti si ha:}
%
{}

\[x_{u-1} = \frac{P - x_u y_{m-u+2}}{y_{m-u+1}}\qquad
  x_{u-1} - y_{m-u+2} = \frac{P - y_{m-n+2} (x_u + y_{m-n+1})}{y_{m-n+1}} > 0\]

\colsni{per le poste ipotesi; e che similmente $y_{m-u}-x_{n+1}>0$.

  Ora supponiamo che alla radice considerata $x_0$ della congruenza
  $\sum C_h x^{n-h} \equiv 0 \mod P$ corrisponda una soluzione propria
  $(x_u,y_{m-u+1})$ dell'equaz. $\sum C_h x^{n-h}\cdot y^h = P$,
  soddisfacente alle condizioni:}
%
{}

\[x_{u+1} \cdot (x_u + y_{m-n+1}) < P\qquad
  y_{m-n+2}\cdot (x_u + y_{m-n+1}) < P\]

\colsni{per quanto precede, essa si troverà in una colonna dello
  specchio~\eqref{S}, e quanto al posto da essa occupato, sarà $x_u$
  il \ordinalnum{1} numero della prima linea di~\eqref{S} che supera
  il corrispondente $x_{m-u+1}$ della \ordinalnum{2}[f], procedendo da
  destra a sinistra, se $x_u>y_{m-u+1}$; oppure, se $x_u<y_{m-u+1}$,
  sarà $y_{m-u+1}$ il \ordinalnum{1} numero della \ordinalnum{2}[f]
  linea di~\eqref{S}, procedendo da sinistra a destra, che supera il
  corrispondente $x_u$ della \ordinalnum{1}[f] linea.

  Concludiamo:

  Se alla radice $x_0$ della congruenza
  $\sum_{h=0}^n C_h x^{n-h}\equiv 0 \mod P$ corrisponde una soluzione
  $(a,b)$ propria dell'equazione $\sum C_h x^{n-h} \cdot y^h = P$,
  tale che $a_1(a+b)<P$, $b_1(a+b)<P$ (essendo $a_1,b_1$ i resti delle
  diivisioni di $C_n b^{n-1}$ per $a$ e rispett. di $C_0 a^{n-1}$ per
  $b$); e meglio ancora (essendo sempre $a_1<a$, $b_1<b$) se la
  soluzione $(a,b)$ è tale che $P$ sia maggiore della maggiore delle
  due quantità $a(a+b)$, $b(a+b)$; allora essa si trova in una colonna
  dello specchio}
%
{}

\begin{equation}
  \tag{\ref{S}}
  \left\{
    \begin{array}{rllllllll}
      x_1 & x_2 & \cdots & x_{u-1} & x_u & x_{u+1} & \cdots & x_{m-1} & x_m = 1\\
      1 = y_m & y_{m-1} & \cdots & y_{m-u+2} & y_{m-u+1} & y_{m-u} & \cdots & y_2 & y_1\\
    \end{array}
  \right.
\end{equation}

\colsni{e precisamente se $y_{m-u+1}$ è il primo termine della
  \ordinalnum{2}[f] linea di~\eqref{S} che supera il corrispondente
  $x_u$ della \ordinalnum{1}[f], procedendo da sinistra a destra,
  questa soluzione sarà $(x_u,y_{m-u+1})$; oppure si troverà nella
  colonna immediatamente a sinistra, cioè sarà $(x_{u-1},y_{m-u+2})$.}
%
{}

\secbreak

\cols{Tutto quanto fin qui si è detto vale per il caso che nella forma
  $\sum_{h=0}^n C_h x^{n-h} \cdot y^h$ il primo e l'ultimo
  coefficiente siano entrambi positivi: agli stessi risultati si
  perviene quando $C_0$ e $C_n$ hanno segni contrarii, p.~es. $C_0>0$,
  $C_n<0$. Solo qui nel procedimento tenuto per risalire da una
  soluzione $(x_0,y_0)$ della congruenza
  $\sum C_h x^{n-h} \cdot y^h\equiv 0 \mod P$ ad una radice della
  $\sum C_h x^{n-h} \equiv 0 \mod P$, bisogna fare le seguenti
  modificazioni. --- Sia dunque}
%
{}

\begin{align*}
  \sum C_h x_0^{n-h} \cdot y_0^h &= 0 \mod P\\
  \sum C_h x_0^{n-h} \cdot y_0^h &= k_0 P.
\end{align*}

\cols{Si divida $C_0 x_0^{n-1}$ per $y_0$, e sia $t_0$ il quoziente,
  $y_\lambda$ il resto, talché:}
%
{}

\[C_0 x_0^{n-1} - t_0 y_0 = y_\lambda\]

\colsni{così si divida $|C_n|y_0^{n-1}$ per $x_0$; e sia $q_0$ il
  quoziente, $z_\lambda$ il resto, di guisa che}
%
{}

\begin{align*}
  &|C_n| y_0^{n-1} - q_0 x_0 = z_\lambda\\
  &q_0 x_0 + C_n y_0^{n-1} = - z_\lambda\\
  &(q_0 + 1)x_0 + C_n y_0^{n-1} = x_0 - z_\lambda = x_\lambda > x_0.
\end{align*}

\cols{Siano $x_1,y_1$ rispett. le radici delle due congruenze:}
%
{}

\begin{align*}
  x_\lambda &\equiv k_0 x \mod x_0\\
  y_\lambda &\equiv k_0 y \mod y_0
\end{align*}

\colsni{talchè}{}

\[k_0 x_1 - x_\lambda = m_0 x_0\qquad
  k_0 y_1 - y_\lambda = n_0 y_0.\]

\cols{Si prova anche qui che la quantità}{}

\[Q = t_0 - (q_0 + 1) + \sum_{h=1}^{n-1} C_h x_0^{n-h-1} \cdot y_0^{h-1} - m_0 - n_0\]

\colsni{è divisibile per $k_0$. --- Infatti}{}

\begin{align*}
  x_0 y_0 Q &= t_0 x_0 y_0 - (q_0 + 1) x_0 y_0 + k_0 P - C_0 x_0^n - C_n y_0^n - m_0 x_0 y_0 - n_0 x_0 y_0\\
            &= k_0 P - k_0(x_1 y_0 + x_0 y_1) = k_0 (P - x_1 y_0 - x_0 y_1) \equiv 0 \mod k_0
\end{align*}

\colsni{laonde, essendo $k_0$ primo con $x_0 y_0$}{}

\[Q \equiv 0 \mod P\]

\colsni{cioè la quantità
  $q_1 = \frac{Q}{k_0} = \frac{P - (x_0 y_1 + x_1 y_0)}{x_0 y_0}$ è un
  intero.

  A questo punto si applica alle coppie $(x_0,x_1)$, $(y_0,y_1)$
  l'algoritmo della ricerca del M.C.D. e si procede in modo al tutto
  identico a quello tenuto per $C_0$ e $C_n$ entrambi positivi,
  pervenendo così alle medesime conclusioni.

  Abbiamo così per ogni caso dato un metodo che permette di trovare le
  soluzioni dell'equazione}
%
{}

\begin{equation}
  \tag{1}\label{main-eq}
  \sum_{h=0}^n C_h x^{n-h} \cdot y^h = P
\end{equation}

\colsni{note che siano le radici della congruenza:}{}

\begin{equation}
  \tag{2}\label{main-eq-mod}
  \sum_{h=0}^n C_h x^{n-h} \equiv 0 \mod P.
\end{equation}

\cols{Vero è però che il metodo delle divisioni successive dà solo le
  soluzioni soddisfacenti alla condizione $2|xy|<P$. Tuttavia si può
  estendere il campo di validità del metodo stesso, qualora si possa
  pensare una trasformazione}
%
{}

\begin{equation*}
  \begin{array}{c}
    x = \alpha x_1 + \beta y_1\\
    y = \gamma x_1 + \delta y_1
  \end{array}
  \quad
  \left|
    \begin{matrix}
      \alpha & \beta\\
      \gamma & \delta
    \end{matrix}
  \right| = \pm 1
\end{equation*}

\colsni{della forma $\sum C_h x^{n-h} \cdot y^h$ in una equivalente,
  per modo che nell'equazione trasformata}
%
{}

\[\sum_{h=0}^n C_h' x_1^{n-h} \cdot y_1^{h} = P\]

\colsni{una eventuale soluzione corrispondente ad una certa radice
  della~\eqref{main-eq-mod}, debba necessariamente soddisfare alla
  condizione $2|x_1y_1|<P$, o anche all'altra più restrittiva}
%
{}

\[x_1(x_1 + y_1) < P\qquad y_1(x_1 + y_1) < P.\]

\cols{In particolare il metodo delle divisioni successive risolve
  completamente l'equaz.~\eqref{main-eq} (se ammette soluzioni),
  quando i $C_h$ sono tutti positivi, quelli di indice dispari nulli,
  ed è $n$ un numero pari.

  Termineremo con un'osservazione sul numero $m$ di queste
  divisioni. --- Riprendiamo i resti $x_1\; x_2\; \ldots\; x_m = 1$
  del sistema di divisioni $(P,x_0)$. Per una nota proposizione di
  aritmetica, si ha:}
%
{}

\begin{gather*}
  x_0 > 2x_2\\
  x_1 > 2x_3\\
  x_2 > 2x_4\\
  \cdots\cdots\\
  x_{m-2} > 2\\
  x_{m-1} \ge 2
\end{gather*}

\colsni{e moltiplicando membro a membro, e sopprimendo i fattori
  comuni:}
%
{}

\begin{align*}
  x_0 x_1 &> 2^m\\
  x_0 (P - x_0) &> 2^m.
\end{align*}

\cols{Riflettendo poi che il valore massimo del prodotto $x_0(P-x_0)$
  si ha per $x_0=\frac{P}{2}$, potremo anche scrivere:}
%
{}

\[2 < \cdots \text{illegible} \cdots.\]

\section*{\selectlanguage{italian}Applicazione del metodo delle
  divisioni successive alla risoluzione dell'equazione indeterminata
  di \ordinalnum{2} grado.}

\cols{Ora vogliamo applicare i risultati ottenuti alla risoluzione
  dell'equazione indeterminata di \ordinalnum{2} grado, ridotta alla
  forma $x^2 \pm qy^2=P$. Condizione necessaria di risolubilità è
  naturalmente l'essere $-q$ residuo quadratico di $P$. Questa
  condizione si supporrà d'ora in avanti sempre soddisfatta, senza
  enunciarlo esplicitamente. Daremo in altra nota delle condizioni
  sufficienti, tra cui

  \begin{enumerate}
  \item $q>0$. --- Se l'equazione $x^2 + q y^2 = p$ è solubile pel più
    piccolo numero primo che dà per residuo $-q$, essa è solubile per
    ogni altro numero primo di cui $-q$ sia resid.\ quad.
  \item $q<0$. --- Se l'equazione $x^2 + q y^2 = p$ è solubile per
    tutti i numeri primi inferiori a $|q$, di cui $-q$ è residuo
    quadrat., essa è solubile per ogni altro numero primo di cui $-q$
    sia r.\ q.
  \item $q>0$ è della forma $4k+1$. --- Se le equaz.\
    $x^2 + q y^2 = 2 p_\lambda$ $x^2 + q y^2 = p_\mu$ sono solubili,
    la \ordinalnum{1}[f] pel più piccolo numero primo della forma
    $4\lambda - 1$ che dà per residuo quad. $-1$, la \ordinalnum{2}[f]
    pel più piccolo numero primo della forma $4\mu+1$ che dà per
    residuo $-q$, e stesse equazioni sono solubili per ogni altro
    numero prima della forma $4\lambda - 1$ e rispett. $4\mu + 1$, di
    cui $-q$ sia resid.\ quadratico.
  \item Se le equazioni $x^2 - q y^2 = 2p_\lambda$
    $x^2 - q y^2 = p_\mu$ dove $q>0$ è della forma $4h-1$, sono
    solubili per tutti i numeri primi della forma $4\lambda-1$ e
    rispett.\ $4\mu+1$ inferiori a $q$, di cui $q$ sia resid. quad., le
    stesse equazioni sono solubili per ogni altro numero primo della
    forma $4\lambda - 1$ e rispett.\ $4\mu + 1$, di cui $+q$ è resid.\
    quad.
  \end{enumerate}

  \paragraph{Risoluzione dell'equazione $x^2 + q y^2 = m$.} Veniamo
  ora ad applicare quanto si è detto sull'equazione
  $\sum_{h=0}^n C_h x^{n-h} \cdot y^h = P$, alla risoluzione
  dell'equaz.\ quadratica
}
%
{}

\begin{equation}
  \tag{1}\label{quadeq}
  x^2 + q y^2 = m.
\end{equation}

\cols{Distinguiamo due casi:
  
  1. $q>0$. --- Se esiste una soluzione intera $(a,b)$
  della~\eqref{quadeq}, sarà
}
%
{}

\[a^2 + q b^2 = m \ge a^2 + b^2 > 2ab\]

\colsni{quindi certamente}{}

\[2ab < m.\]

\cols{Il metodo delle divisioni successive dà dunque tutte le
  soluzioni proprie, se esistono dell'eq.~\eqref{quadeq}.

  Se $x_0 > \frac{m}{2}$ è una radice della congruenza}
%
{}

\begin{equation}
  \tag{2}\label{quadeqmod}
  x^2 + q \equiv 0 \mod m
\end{equation}

\colsni{e $x_1, x_2, x_3, \ldots, x_r=1$ la successione dei resti,
  $q_0=1, q_1, q_2, \ldots, q_{r-1}$ quella dei quozienti del sistema
  di divisioni $(m,x_0)$, fatto $q_r = x_{r-1}-1$, si ponga}
%
{}

\begin{equation*}
  y_r=1\quad y_{r-1} = |1,q_1|\quad y_{r-2} = |1,q_1,q_2|,\quad \ldots,\quad
  y_1 = |1,q_1,q_2,\ldots,q_{r-1}|,\quad
  y_0 = |1,q_1,q_2,\ldots,q_{r-1},q_r|.
\end{equation*}

\cols{Come sappiamo, è allora $x_0 y_0\equiv \pm 1 \mod m$, ed
  $y_1, y_2, \ldots, y_r=1$ i resti nel sistema di divisioni
  $(m,y_0)$.

  Si formi lo specchio
}

\begin{equation}
  \tag{S}\label{S}
  \left\{
    \begin{array}{rllllllll}
      x_1 & x_2 & \cdots & x_{u-1} & x_u & x_{u+1} & \cdots &  x_{r-1} & x_r = 1  \\
      1 = y_r & y_{r-1} & \cdots & y_{r-u+2} & y_{r-u+1} & y_{r-u} & \cdots & y_2 & y_1 
    \end{array}
  \right.
\end{equation}

\colsni{nel quale i resti della \ordinalnum{1}[f] linea sono ordinati
  in ordine decrescente, quelli della \ordinalnum{2}[f] in ordine
  crescente. --- Alla radice considerata $x_0$ corrisponda una
  soluzione (propria) dell'equazione~\eqref{quadeq}, e sia essa la
  $(x_u,y_{r-u+1})$. Si ha qui}
%
{}

\[y_{r-u+2} = x_u - t_0 y_{r-u+1}\qquad
  x_{u+1} = q y_{r-u+1} - q_0 x_u\]

\colsni{dove $q_0, t_0$ sono interi positivi o nulli. --- Allora}{}

\begin{align*}
  x_{u+1} ( x_u + y_{r-u+1}) &= (q y_{r-u+1} - q_0x_u) ( x_u + y_{r-u+1}) =\\
                             &= q y_{r-u+1}^2 + q x_u y_{r-u+1} - q_0 x_u^2 - q_0 x_u y_{r-u+1} =\\
                             &= m - x_u^2 + q x_u y_{r-u+1} - q_0 x_u^2 - q_0 x_u y_{r-u+1} =\\
                             &= m - x_u (x_u + q_0 x_u + q_0 y_{r-u+1} - q y_{r-u+1}) =\\
                             &= m - x_u (x_u + q_0 y_{r-u+1} - x_{u+1}) < m
\end{align*}

\cols{Similmente}{}

\begin{multline*}
  y_{r-u+2} (x_u + y_{r-u+1})
  = (x_u - t_0 y_{r-u+1}) (x_u + y_{r-u+1})
  = m - q y_{r-u+1}^2 - t_0 x_u y_{r-u+1} - t_0 y_{r-u+1}^2 + x_u y_{r-u+1} =\\
  = m - y_{r-u+1} (q y_{r-u+1} + t_0 x_u - y_{r-u+2}) < m
\end{multline*}

\cols{Concludiamo:}

\[x_{u+1} (x_u + y_{r-u+1}) < m\qquad y_{r-u+2} (x_u + y_{r-u+1}) < m\]

\cols{applicando perciò un risultato precedentemente ottenuto, avremo
  che $x_u$ è il primo termine della \ordinalnum{1}[f] linea dello
  specchio precedente~\eqref{S} che supera il corrispondente
  $y_{r-u+1}$ della \ordinalnum{2}[f], (procendendo da destra a
  sinistra, quando sia $x_u>y_{r-u+1}$: mentre, se $y_{r-u+1}>x_u$,
  sarà $y_{r-u+1}$ il primo termine della \ordinalnum{2}[f] linea
  di~\eqref{S} che supera il corrispondente $x_u$ della
  \ordinalnum{1}[f], procedendo da sinistra a destra. --- Per mezzo
  dello specchio~\eqref{S} si può dunque trovar subito, se esiste, la
  soluzione $(x_u,y_{r-u+1})$ dell'equaz.\ $x^2+q y^2 = m$
  corrispondentemente a una data radice $x_0$ della congr.}
%
{}

\[x^2 + q = 0 \mod m.\]

\cols{In particolare, se $q=1$, le $x$ coincidono con le $y$, e si
  ritrova ciò che si era già dimostrato, che cioè la soluzione
  dell'eq.\ $x_2 + y^2 = m$, corrispondente alla radice $x_0$ della
  cong. $x^2+1\equiv 0 \mod m$, è data dalla coppia dei resti
  intermedii nel sistema di divisioni $(m,x_0)$.

  \emph{Es.} Sia l'equazione $x^2 + y^2 = 89$. La radice
  $>\frac{89}{2}$ della cong.\ $x^2+1=0\mod 89$ è $55$. Sviluppiamo il
  sistema di divisioni $(89,55)$
}
%
{}

\[\frac{89}{34}^\frac{|55}{1}\quad
  \frac{55}{21}^\frac{|34}{1}\quad
  \frac{34}{13}^\frac{|21}{1}\quad
  \frac{21}{8}^\frac{|13}{1}\quad
  \frac{13}{5}^\frac{|8}{1}\quad
  \frac{8}{3}^\frac{|5}{1}\quad
  \frac{5}{2}^\frac{|3}{1}\quad
  \frac{3}{1}^\frac{|2}{1}\]
   
\colsni{i resti intermedii sono $8$ e $5$, e si ha appunto}{}

\[8^2 + 5^2 = 89.\]

\cols{Sia l'equazione $x^2 + 13 y^2 = 3221$: la radice della
  congruenza $x^2+13=0\mod 3221$ è $x_0=2723$. Il sistema di divisioni
  $(3221,2723)$ è}
%
{}

\[\frac{3221}{498}^\frac{|2723}{1}\quad
  \frac{2723}{233}^\frac{|498}{5}\quad
  \frac{498}{32}^\frac{|233}{2}\quad
  \frac{233}{9}^\frac{|32}{7}\quad
  \frac{32}{5}^\frac{|9}{3}\quad
  \frac{9}{4}^\frac{|5}{1}\quad
  \frac{5}{1}^\frac{|4}{1}.\]

\cols{Formando successivamente i numeri}{}

\begin{gather*}
  1,\; [1,5] = 6,\; [1,5,2] = 13,\; [1,5,2,7] = 97,\; [1,5,2,7,3]=301,\\
  [1,5,2,7,3,1]=401,\; [1,5,2,7,3,1,1]=705,
\end{gather*}

\colsni{e con questi lo specchio}{}

\[
  \begin{array}{ccccccc}
    498 & 233 & 32 & 9 & 5 & 4 & 1\\
    1 & 6 & 13 & 97 & 304 & 401 & 705
  \end{array}\]

\colsni{vediamo che il primo resto della \ordinalnum{2}[f] linea che
  supera il corrispondente della \ordinalnum{1}[f], è $97$: perciò la
  soluzione cercata sarà $(9,97)$, oppure $(32,13)$. --- Si trova che
  è appunto}
%
{}

\[32^2 + 13 \cdot 13^2 = 3221.\]

\cols{Facciamo un'ultima osservazione sulla risoluzione dell'equazione
  $x^2 + q y^2=m (q>0)$. --- Essendo sempre $x_0 > \frac{m}{2}$ una
  radice della congruenza $x^2+q\equiv 0 \mod m$, ed
  $y_0 > \frac{m}{2}$ il numero $<m$ tale che
  $x_0 y_0 \equiv \pm 1 \mod m$, siano $x_1,x_2,\ldots,x_n=1$;
  $y_1,y_2,\ldots,y_n=1$ le due successioni dei resti nei sistemi di
  divisioni $(m,x_0)$, $(m,y_0)$ (naturalmente abbiamo supposto $x_0$
  e quindi $y_0$ primo con $m$). --- Disponiamo le due successioni una
  sotto l'altra a questo modo:}
%
{}

\begin{equation}
  \tag{\ref{S}}
  \left\{
    \begin{array}{rllllll}
      x_1& x_2 & \cdots & x_r & x_{r+1} & \cdots& x_n = 1\\
      y_n = 1 & y_{n-1} & \cdots & y_{n-r+1} & y_{n-r} & \cdots & y_1
    \end{array}
  \right.
\end{equation}

\colsni{se $y_{n-r+1}$ è il primo termine della \ordinalnum{2}[f]
  linea che supera il corrispondente $x_r$ della \ordinalnum{1}[f], e
  se alla radice considerata $x_0$ corrisponde una soluzione propria
  all'equaz.\ $x^2+q y^2 = m$, essa sarà data da $(x_r,y_{n-r_1})$,
  ovvero da $(x_{r-1},y_{n-r+2})$. --- Ora qui dimostreremo che si può
  fare a meno della \ordinalnum{2}[f] successione
  $y_1,y_2,\ldots,y_n$; e persino \emph{[illegible]}

  Sia infatti $(x_\lambda,y_{n-\lambda+1})$ la soluzione (supposta
  esistente) dell'eq.\ $x^2+qy^2=m$ corrispondente alla radice
  considerata $x_0$ della congruenza $x^2+q\equiv 0 \mod m$. --- Si ha
  intanto}

\[x_\lambda^2 + qy_{n-\lambda+1}^2 = m\]

\colsni{e quindi}{}

\[x_\lambda^2<m.\]

\cols{Ora dimostreremo che}{}

\[x_{\lambda-1}^2 > m.\]

\cols{Si ha infatti}{}

\[x_{\lambda-1} = t_0 x_\lambda + q y_{n-\lambda+1}\]

\colsni{dove $t_0$ è il quoziente della divisione di $x_\lambda$ per
  $y_{n-\lambda+1}$.

  Se $t_0 = 0$, si ha}

\[x_{\lambda-1} = q y_{n-\lambda+1}\qquad
  x_{\lambda-1}^2 = q(q y_{n-\lambda+1}^2)\]

\colsni{e quindi, essendo in tal caso $y_{n-\lambda+1}>x_\lambda$, si
  ha certamente (almeno per $q>1$)}
%
{}

\[x_{\lambda-1}^2 = q (q y_{n-\lambda+1}^2) > x_\lambda^2 + q y_{n-\lambda+1}^2 = m.\]

\cols{Se invece $t_0$ è diverso da zero, allora dall'espressione di
  $x_{\lambda-1}$, essendo $x_\lambda$, $y_{n-\lambda+1}$, $q$, $t_0$
  quantità positive, risulta evidentemente}
%
{}

\[x_{\lambda-1}^2 > m.\]

\cols{Dunque effettivamente $x_\lambda$ è il \ordinalnum{1} resto nel
  sistema di divisioni $(m,x_0)$, il cui quadrato non super $m$. ---
  Concludiamo:

  \guillemotleft{} Se alla radice considerata $x_0$ della cong.\
  $x^2+q\equiv 0 \mod m$ corrisponde una soluzione propria
  $(x_\lambda,y_{n-\lambda+1})$ dell'equazione $x^2+qy^2=m$, il primo
  termine $x_\lambda$ di essa è dato dal \ordinalnum{1} resto, nel
  sistema di divisioni $(m,x_0)$ il cui quadrato non supera $m$. ---
  Se pertanto questo primo resto $x_\lambda$ è tale che
  $m-x_\lambda^2$ non sia divisibile per $q$, o se anche, essendo
  divisibile per $q$, non risulta $\frac{m-x_\lambda^2}{q}$ un
  quadrato perfetto, si è certi che alla radice $x_0$ non corrisponde
  alcuna soluzione (propria) dell'eq.\ $x^2+qy^2=m$ \guillemotright{}.

  \emph{Es.} 1. Sia l'equazione $x^2+89y^2=1171$: la radice della
  cong.\ $x^2+89\equiv 0 \mod 1171$ è $x_0=752$. Il sistema di
  divisioni $(1171,752)$ è}
%
{}

\[\frac{1171}{419}^\frac{|752}{1}\quad
  \frac{752}{333}^\frac{|419}{1}\quad
  \frac{419}{86}^\frac{|333}{1}\quad
  \frac{333}{75}^\frac{|86}{3}\quad
  \frac{86}{11}^\frac{|75}{1}\quad
  \cdots\]

\cols{Il \ordinalnum{1} resto, il cui quadrato non supera $1171$ è
  $11$. Ora $1171-11^2=1050$ che non è divisibile per $89$: tanto
  basta per affermare che alla radice considerata $x_0=752$ non
  corrisponde nessuna soluzione dell'equaz.\ $x^2+89y^2=1171$.

  2. Sia l'equazione $x^2+29y^2=2489$: una radice della congruenza
  $x^2+29\equiv 0\mod 2489$ è $x_0=1878$. --- Il sistema di divisioni
  $(2489,1878)$ dà:}
%
{}

\[\frac{2489}{611}^\frac{|1878}{1}\qquad
  \frac{1878}{45}^\frac{|611}{3}\qquad
  \cdots\]

\cols{Il \ordinalnum{2} resto $45$ è già tale che il suo quadrato $2025$ non supera $2489$.

  Dunque se alla radice considerata $x_0=1878$ corrisponde una
  soluzione dell'equaz.\ $x^2+29y^2=2489$, il suo primo termine $x$
  non potrà essere che $45$. In effetto si ha}
%
{}

\[2489 - 45^2 = 29 \cdot 16\qquad
  \frac{2489 - 45^2}{29} = 4^2;\]

\colsni{perciò $45^2+29\cdot 4^2 = 2489$. Alla radice considerata
  corrisponde dunque la soluzione $(45,4)$.

  --- Come si vede dagli esempii dati, se $x_0>\frac{m}{2}$ è una
  radice della congruenza $x^2+q\equiv 0 \mod m$, per decidere se ad
  essa radice corrisponda una soluzione (propria) dell'eq.\
  $x^2 + qy^2=m$, non è nemmeno necessario sviluppare tutto il sistema
  di divisioni $(m,x_0)$; ma basta fermarsi al primo resto, il cui
  quadrato non supera $m$.

  \paragraph{II Caso.} Nell'equazione $x^2 + qy^2 = m$ sia $q<0$:
  possiamo scrivere senza altro sotto la forma}

\begin{equation}
  \tag{1}\label{quadeq-neg}
  x^2 - q y^2 = m
\end{equation}

\colsni{dove supponiamo $q>0$, (naturalmente non quadrato perfetto).

  Sappiamo che è sempre solubile l'equazione}
%
{}

\begin{equation}
  \tag{2}\label{quadeq-neg-1}
  x^2 - q y^2 = 1.
\end{equation}

\cols{Ne sia $(\alpha,\beta)$ la soluzione minima. Allora:}
{}

\[\alpha^2 - q \beta^2 = 1\qquad
  (\alpha^2 - q \beta^2)^2 = (\alpha^2 + q \beta^2)^2 - q(2\alpha\beta)^2 = 1\]

\colsni{cioè simbolicamente}{}

\[(\alpha,\beta)^2 = (\alpha^2 + q\beta^2, 2\alpha\beta)\]

\colsni{così}{}

\[(\alpha,\beta)^3 = \Bigl(\alpha (\alpha^2 + 3q \beta^2), \beta (3\alpha^2 + q\beta^2) \Bigr)\]

\colsni{in generale}{}

\begin{align*}
  (\alpha, \beta)^n = \biggl[ &\left(\alpha^n + \binom{n}{2}\alpha^{n-2} \cdot \beta^2 q
                               + \binom{n}{4}\alpha^{n-4} \cdot \beta^4 q^2 
                               + \binom{n}{6}\alpha^{n-6} \cdot \beta^6 q^3
                               + \cdots \right),\\
                             &\left(\binom{n}{1}\alpha^{n-1} \cdot \beta
                               + \binom{n}{3}\alpha^{n-3} \cdot \beta^3 q
                               + \binom{n}{5}\alpha^{n-5} \cdot \beta^5 q^2
                               + \cdots \right)
                               \biggr].
\end{align*}

\cols{È noto che le soluzioni}{}

\[(\alpha,\beta)^1, (\alpha,\beta)^2, (\alpha,\beta)^3, \ldots, (\alpha,\beta)^n, \ldots\]

\colsni{ottenute in questo modo, sono \emph{tutte} le soluzioni
  della~\eqref{quadeq-neg-1}, disposte in ordine di grandezza crescente.

  Rivolgendoci ora all'equazione~\eqref{quadeq-neg}, si sa che se
  (corrispondentemente a una determinata radice della congruenza}
%
{}

\begin{equation}
  \tag{3}\label{quadeq-neg-mod}
  x^2 - q \equiv 0 \mod m )
\end{equation}

\colsni{ammette una soluzione, ne ammette infinite. Sia $(a,b)$ la
  soluzione minima, corrispondente alla radice $x_0$
  della~\eqref{quadeq-neg-mod}.

  Se si pone}
%
{}

\[(\alpha,\beta)^k = (\alpha_k,\beta_k),\]

\colsni{dalle}{from}

\[\alpha_k^2 - q \beta_k^2 = 1\qquad
  a^2 - q b^2 = m\]

\colsni{si deduce}{}

\[(\alpha_k^2 - q\beta_k^2) (a^2 - qb^2)
  = (a\alpha_k \pm qb_k^2)^2 - q (a\beta_k \pm b\alpha_k)^2 = m\]

\colsni{cioè simbolicamente}{}

\begin{align*}
  (\alpha, \beta)^k \cdot (a,b) &= \{ a\alpha_k + qb\beta_k, a\beta_k + b\alpha_k \}\\
  (\alpha, \beta)^{-k} \cdot (a,b) &= \{ a\alpha_k - qb\beta_k, a\beta_k - b\alpha_k \}.
\end{align*}

\cols{Ora è facile vedere che ponendo successivamente}{}

\begin{gather*}
  (a,b) = (a,b),\qquad
  (\alpha,\beta)^{-1} \cdot (a,b) = (a_1,b_1),\qquad
  (\alpha,\beta)^2 \cdot (a,b) = (a_2, b_2),\\
  (\alpha,\beta)^{-2} \cdot (a,b) = (a_3,b_3),\qquad
  (\alpha,\beta)^2 \cdot (a,b) = (a_4,b_4),\qquad \ldots,\\
  (\alpha,\beta)^{-k} \cdot (a,b) = (a_{2k-1},b_{2k-1}),\qquad
  (\alpha,\beta)^k \cdot (a,b) = (a_{2k}, b_{2k}),\qquad, \ldots
\end{gather*}

\colsni{si ottengono tutte le soluzioni della~\eqref{quadeq-neg}
  (appartenenti alla radice considerata $x_0$), e che di più sono
  disposte in ordine di grandezza crescente. --- Ma ritornando alla
  soluzione minima $(a,b)$, troviamo le proprietà che la distinguono
  dalle altre soluzioni.

  Si ha}
%
{}

\[(a_1,b_1) = (a\alpha - qb\beta, a\beta - b\alpha)\]

\colsni{e per l'ipotesi che $(a,b)$ sia la minima}{}

\[|a\alpha - qb\beta| > a\qquad |a\beta-b\alpha|>b.\]

\cols{Se fosse}{}

\[qb\beta - a\alpha > a\qquad b\alpha - a\beta > b\]

\end{document}

% Local Variables:
% ispell-local-dictionary: "italian"
% End:
