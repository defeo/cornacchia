\documentclass[12pt,leqno]{article}

\usepackage[a4paper,margin=1in,landscape]{geometry}
\usepackage[utf8]{inputenc}
\usepackage[T1]{fontenc}
\usepackage{parcolumns}
\usepackage[italian,american]{babel}
\usepackage{amsmath,amssymb,amsthm}
\usepackage{fmtcount}

\newtheorem{coroen}{Corollary}
\newtheorem{coroit}{Corollario}
\newenvironment{corollary}
{\iflanguage{italian}{\coroit}{\coroen}}
{\iflanguage{italian}{\endcoroit}{\endcoroen}}

\newcommand{\cols}[2]{\begin{parcolumns}[rulebetween]{2}%
    \selectlanguage{italian}\colchunk{#1}%
    \selectlanguage{american}\colchunk{#2}%
  \end{parcolumns}}
\newcommand{\colsni}[2]{\begin{parcolumns}[rulebetween,nofirstindent]{2}%
    \selectlanguage{italian}\colchunk{#1}%
    \selectlanguage{american}\colchunk{#2}%
  \end{parcolumns}}

\newcommand{\footnotecols}[2]{\footnotetext{%
    \begin{minipage}{0.95\linewidth}\cols{#1}{#2}\end{minipage}%
}}

\newcommand\secbreak{%
  \par\bigskip\noindent\hfill\rule{15em}{0.1px}\hfill\null\par\bigskip
}

\title{Su di un metodo per la risoluzione in numeri interi
  dell'equazione\\
  $\sum_{h=0}^nC_hx^{n-h}\cdot y^h=P$}
\author{Dott. Giuseppe Cornacchia\\[1em]
  translation by Luca De Feo}

\begin{document}
\maketitle

\cols{Il metodo che andiamo ad esporre si fonda sul noto
  \emph{algoritmo di Eulero} che si applica alla risoluzione
  dell'equazione $ax+by=1$.
  
  Se si hanno due quantità indeterminate qualunque $a,b,$ e una serie
  di altre quantità $\gamma,\delta,\varepsilon,\ldots,\lambda,\mu,\nu$,
  e si forma con esse una nuova serie di quantità $c,d,e,\ldots,l,m,n$
  operando secondo la legge seguente:}
{The method we're going to explain is funded on Euler's
  well known algorithm, which applies to the solution of the
  equation $ax+by=1$.
  
  If we have two arbitrary indeterminate quantities $a,b,$ and a
  series of other quantities
  $\gamma,\delta,\varepsilon,\ldots,\lambda,\mu,\nu$, and we form with
  them a new series of quantities $c,d,e,\ldots,l,m,n$ acting by the
  following law:}

\[c = \gamma b + a\quad
  d = \delta c + b\quad
  e = \varepsilon d + e,\ldots,
  n = \nu m + l\]

\colsni{si potranno ottenere tutte le quantità $c,d,e,\ldots,l,m,n,$
  successivamente, in funzione di $a$ e $b$. In particolare:}
%
{we can obtain all quantities  $c,d,e,\ldots,l,m,n,$
  successively, as a function of $a$ e $b$. In particular:}

\[n = Ga+Hb\]

\colsni{dove $G,H$ sono funzioni di $\gamma,\delta,\varepsilon,\ldots$
  indipendenti da $a$ e $b$. Usando il simbolo di Gauss, si ha}
%
{where $G,H,$ are function of $\gamma,\delta,\varepsilon,\ldots$
  independent from $a$ and $b$. Using Gauss' symbol, we have}

\[H = [\gamma,\delta,\varepsilon,\ldots,\lambda,\mu,\nu]\quad
  G=[\delta,\varepsilon,\ldots,\lambda,\mu,\nu].\]

\cols{Le espressioni $H$ obbediscono a due semplici leggi di
  formazione, l'una procedendo verso destra, l'altra verso sinistra,
  che sono espresse dalle formule:}
%
{The expressions $H$ obey two simple laws, one proceding from the
  right, the other from the left, as expressed by the formulas:}
%
\begin{align*}
  [\gamma,\delta,\varepsilon,\ldots,\lambda,\mu,\nu] &=
  \gamma[\delta,\varepsilon,\ldots,\lambda,\mu,\nu] +
  [\varepsilon,\ldots,\lambda,\mu,\nu]\\
  [\gamma,\delta,\varepsilon,\ldots,\lambda,\mu,\nu] &=
  [\gamma,\delta,\varepsilon,\ldots,\lambda,\mu]\nu +
  [\gamma,\delta,\varepsilon,\ldots,\lambda].
\end{align*}

\cols{Si ha poi}{Then we have}

\[[\gamma,\delta,\varepsilon,\ldots,\lambda,\mu,\nu] =
  [\nu,\mu,\lambda,\ldots,\varepsilon,\delta,\gamma];\quad
  [-\gamma,-\delta,-\varepsilon,\ldots,-\lambda,-\mu,-\nu] =
  \pm[\gamma,\delta,\varepsilon,\ldots,\lambda,\mu,\nu].\]

\cols{Queste espressioni sono della massima importanza nella teoria
  delle frazioni continue. Infatti, se una frazione continua
  ordinaria, i cui numeratori sono tutti uguali ad 1, si rappresenta
  col simbolo
  $(\gamma,\delta,\varepsilon,\ldots,\lambda,\mu,\nu,\ldots)$ talché
  sia}
%
{}

\[(\gamma,\delta,\varepsilon,\ldots,\lambda,\mu,\nu) =
  \gamma+\frac{1}{(\gamma,\delta,\varepsilon,\ldots,\lambda,\mu,\nu)} =
  \left(\gamma,\delta,\varepsilon,\ldots,\lambda,\mu+\frac{1}{\nu}\right)\]

\colsni{si ottiene in generale, mediante riduzione:}{}

\[(\gamma,\delta,\varepsilon,\ldots,\lambda,\mu,\nu) =
  \frac{[\gamma,\delta,\varepsilon,\ldots,\lambda,\mu,\nu]}
  {[\delta,\varepsilon,\ldots,\lambda,\mu,\nu]}.\]

\cols{Se i numeri $\gamma,\delta,\varepsilon,\ldots,\lambda,\mu,\nu$
  sono interi, altrettanto può dirsi dei numeratori e dei denominatori
  delle frazioni}
%
{}

\[\frac{[\gamma]}{[1]}, \frac{[\gamma,\delta]}{[\delta]},
  \frac{[\gamma,\delta,\varepsilon]}{[\gamma,\delta]}, \ldots,
  \frac{[\gamma,\delta,\varepsilon,\ldots,\lambda,\mu,\nu]}
  {[\delta,\ldots,\lambda,\mu,\nu]},\]

\colsni{ognuna delle quali è inoltre irreducibile.

  Vediamo qualche altra propriétà delle espressioni sopra
  accennate. ---Se, prese delle quantità $1,q_1,q_2,\ldots,q_{n-1},q_n$
  si formano successivamente le espressioni:}
%
{}
%
\begin{gather*}
  y_{n+1}=1\quad y_n=1\quad y_{n-1}=[1,q_1]\quad y_{n-2}=[1,q_1,q_2),\ldots\\
  y_1=[1,q_1,q_3,\ldots,q_{n-1}]\quad;\quad
  y_0=[1,q_1,q_2,\ldots,q_{n-1},q_n]
\end{gather*}

\colsni{e poi le epressioni}{}
%
\begin{gather*}
  x_{n+1}=1\quad x_n=1\quad x_{n-1}=[1,q_n]\quad x_{n-2}=[1,q_n,q_{n-1}],\ldots\\
  x_1=[1,q_n,q_{n-1},\ldots,q_2]\quad;\quad
  x_0=[1,q_n,q_{n-1},\ldots,q_2,q_1]  
\end{gather*}

\colsni{dico che si ha:}{}
%
\begin{align}
  &x_r\cdot y_{n-r} + x_{r+1}\cdot y_{n-r+1} = P
  &(r=0,1,2,\dots,n)
  \tag{\(\alpha\)}\label{alpha}\\
  &x_r\cdot y_s + (-1)^{n+r+s}\cdot x_{n-s+1}\cdot y_{n-r+1} \equiv 0 \mod P
  &\left(\begin{array}{c}r=0,1,\dots,n\\s=0,1,\dots,n\end{array}\right)
  \tag{\(\beta\)}\label{beta}
\end{align}

\colsni{avendo posto $P=x_0+x_1=y_0+y_1$. La \eqref{beta} vale
  naturalmente solo quando le $q$ sono numeri interi. ---Per la legge
  esposta di formazione, si ha:}
%
{}

\[x_{r+1} = x_{r-1} - q_r x_r \quad y_{n-r}=q_r y_{n-r+1} + y_{n-r+2}\]

\colsni{posto}{}

\[q_0 = 1 \quad x_{-1} = x_0 + x_1 = P.\]

\cols{Sostituendo nel \ordinalnum{1} membro della \eqref{alpha},
  avremo:}{}

\[x_r\cdot y_{n-r} + x_{r+1}\cdot y_{n-r+1} =
  x_r(q_r y_{n-r+1} + y_{n-r+1}) + y_{n-r+1}(x_{r-1} - q_r x_r) =
  x_{r-1}\cdot y_{n-r+1} + x_r\cdot y_{n-r+2}\]

\cols{Dunque l'espressione $x_r\cdot y_{n-r} + x_{r+1}\cdot y_{n-r+1}$
  non cambia valore mutando $r$ in $r-1$, e quindi neppure cambiando
  $r$ in $r-\lambda (\lambda=0,1,2,\dots r)$.  Laonde:}
%
{}

\[x_r\cdot y_{n-r} + x_{r+1}\cdot y_{n-r+1} =
  x_{r-\lambda}\cdot y_{n-(r-\lambda)} + x_{(r-\lambda)+1}\cdot y_{n-(r-\lambda)+1}.\]

\colsni{e per $r=\lambda$ si ha}{}

\[x_r\cdot y_{n-r} + x_{r+1}\cdot y_{n-r+1} = x_0 + x_1 = y_0 + y_1 = P.\]

\cols{Così è dimostrata la \eqref{alpha}; dalla \eqref{beta} si possono
  come casi particolari estrarre le due congruenze}
%
{}
%
\begin{align}
  x_r + (-1)^rx_1\cdot y_{n-r+1} &\equiv 0 \mod P
  \tag{\(\beta_1\)}\label{beta1}\\
  y_s + (-1)^s\cdot y_1 \cdot x_{n-s+1} &\equiv 0 \mod P
  \tag{\(\beta_2\)}\label{beta2}
\end{align}
 
\colsni{le quali intanto sussistono, come si verifica subito, la
  \ordinalnum{1}[f] per $r=0,1$; la \ordinalnum{2}[f] per
  $s=0,1$. ---E col solito metodo da $r$ a $r+1$ (o da $s$ ad $s+1$)
  valendosi delle relazioni che legano le quantità $x$ e le $y$, si
  dimostra che sono vere per qualunque valore degli indici $r,s$
  rispettivamente.--- Le \eqref{beta1}, \eqref{beta2} si possono anche
  scrivere:}
%
{}
%
\begin{align}
  x_r &\equiv - (-1)^rx_1\cdot y_{n-r-1} \mod P
  \tag{\(\beta_1'\)}\label{beta1'}\\
  y_s &\equiv - (-1)^sy_1\cdot x_{n-s-1} \mod P
  \tag{\(\beta_2'\)}\label{beta2'}
\end{align}

\colsni{e moltiplicando membro a membro}{}

\[x_r\cdot y_s \equiv (-1)^{r+s}\cdot x_1\cdot y_1\cdot y_{n-r+1}\cdot x_{n-s+1}.\]

\cols{Ma la \ref{beta1'} dà per $r=n$}{}

\[1 \equiv - (-1)^n x_1 \cdot y_1\]

\colsni{e però, come si voleva:}

\[x_r\cdot y_s + (-1)^{n+r+s}\cdot x_{n-s+1}\cdot y_{n-r+1} \equiv 0 \mod P.\]

\cols{Ciò posto, sia $P$ un intero qualunque, ed $a$ un numero primo
  con $P$ e minore di $P$ ma maggiore di $\frac{1}{2}P$. Si applichi
  alla coppia $(P,a)$ l'algoritmo della ricerca del M.C.D.; per
  l'ipotesi posta si perverrà al resto $1$ dopo un numero finito $n$
  di divisioni. Sia $q_0=1,q_1,q_2,\ldots,q_{n-1}$ la serie dei
  quozienti; $x_1,x_2,\ldots,x_n=1$ la serie dei resti corrispondenti.

  Posto $P=x_{-1},a=x_0,x_{n+1}=1,q_n=x_{n-1}-1$ avremo:}
%
{}

\begin{equation}
  \label{1}
  x_{i-1} = q_i x_i + x_{i+1}.\qquad
  (i=0,1,\ldots, n)
\end{equation}

\cols{Si ponga ora il sistema}{}

\begin{equation}
  \label{2}
  y_{n-i} = q_i y_{n-i+1} + y_{n-i+2}\qquad
  (i=0,1,\ldots, n)
\end{equation}

\colsni{a cui si aggiungano le condizioni $y_n=y_{n+1}=1$.

  Si avrà allora, introducendo il simbolo di Gauss:}

\[y_{n-i} = [1,q_1,q_2,\ldots,q_i]\qquad
  x_{i-1} = [1,q_n,q_{n-1},\ldots,q_i].\]

\cols{Richiamando le propriétà sopra dimostrate, avremo:}{}

\begin{align}
  &x_r\cdot y_{n-r} + x_{r+1}\cdot y_{n-r+1} = x_0 + x_1 = y_0 + y_1 = P
  &(r=0,1,2,\dots,n)
  \tag{\ref{alpha}}\\
  &x_r\cdot y_s + (-1)^{n+r+s}\cdot x_{n-s+1}\cdot y_{n-r+1} \equiv 0 \mod P
  &\left(\begin{array}{c}r=0,1,\dots,n\\s=0,1,\dots,n\end{array}\right)
  \tag{\ref{beta}}
\end{align}

\colsni{da cui in particolare per $r=s=1$}{}

\[x_1\cdot y_1 \equiv (-1)^{n-1} \mod P\]

\colsni{e ancora}{and}

\[x_0\cdot y_0 \equiv (-1)^{n-1} \mod P.\]

\cols{Si è trovata una soluzione $y=y$ della congruenza}{}

\[ay \equiv \pm 1 \mod P.\]

\cols{Ponendo mente alle relazioni ricorrenti~\eqref{2} e ricordando
  che $y_0+y_1=P$, si scorge come i numeri $y_1,y_2,\dots,y_n=1$ sono
  i resti delle successive divisioni che si ottengono applicando ai
  numeri $(P,y_0)$ l'algoritmo della ricerca del M.C.D. Si ha poi
  $x_0>\frac{1}{2}P,y_0>\frac{1}{2}P$. Se $n$ è dispari, allora
  $x_0 y_0 \equiv 1\mod P$, cioè la radice della congruenza}
%
{}

\[x_0 y \equiv 1 \mod P\]

\colsni{è maggiore di $\frac{1}{2}P$; se $n$ è pari, si ha
  $x_0y\equiv -1, x_0 y_1 \equiv 1 \mod P$; in questo caso la radice
  $y_1$ della congruenza $x_0y\equiv 1 \mod P$ è $<\frac{1}{2}P$. Dopo
  ciò possiamo enunciare:

  \guillemotleft Dato un intero qualunque $P$ e un numero
  $x_0<\frac{1}{2}P$, primo con $P$ e inferiore ad esso, sia $y_0$ la
  radice della congruenza}
%
{}

\[x_0 y \equiv 1\mod P\]

\colsni{in altri termini, sia $y_0$ il \emph{coniugato} di
  $x_0\mod P$: applicando ai numeri $(P,x_0)$ l'algoritmo della
  ricerca del M.C.D. si perviene al resto $1$ dopo un numero $n$ di
  divisioni pari o dispari, secondo che
  $y_0\lessgtr\frac{1}{2}P$. ---In ogni caso, applicando lo stesso
  procedimento ai numeri $(P,y_0)$
  $\left\{ \text{oppure } (P,P-y_0) \text{ se } y_0<\frac{1}{2}P
  \right\}$ si perviene al resto $1$ dopo lo stesso numero $n$ di
  divisioni.--- Le due successioni dei resti e dei quotienti godono le
  seguenti proprietà:

  Nei due sistemi di divisioni il \ordinalnum{2} quoziente dell'un
  sistema è uguale al penultimo resto dell'altro, diminuito di 1. ---I
  quozienti della \ordinalnum{1}[f] successione, a cominciare dal
  terzo, si riproducono nella \ordinalnum{2}[f] in ordine inverso.

  La somma dei prodotti in croce di due resti consecutivi del
  \ordinalnum{1} sistema, e dei corrispondenti nel \ordinalnum{2}
  occupanti i posti complementari ad $n+1$ (cioè se i primi due resti
  occupano i posti $t,t+1$, i corrispondenti occuperanno i posti
  $n-t,n-t+1$) è uguale a $P$.

  Presi due resti qualunque $x_r,y_s$, l'uno del \ordinalnum{1}
  sistema, l'altro del \ordinalnum{2}, e i loro rispettivi
  complementari $x_{n-s+1},y_{n-r+1}$, si ha}
%
{}

\[x_r\cdot y_s + (-1)^{n+r+s}\cdot x_{n-s+1}\cdot y_{n-r+1}\equiv 0 \mod P.\]

\cols{In particolare, presi due resti $x_r,x_{n-r+1}$ del
  \ordinalnum{1} sistema equidistanti dagli estremi, e i
  corrispondenti $y_r,y_{n-r+1}$ nel \ordinalnum{2}, si ha:}
%
{}

\[x_r\cdot y_r + (-1)^n x_{n-r+1}\cdot y_{n-r+1}\equiv 0 \mod P.\]

\cols{Vediamo ora due casi particolari notevoli. Supponiamo in primo
  luogo che si abbia}
%
{}

\[x_0^2 \equiv 1 \mod P\]

\colsni{cioé sia $x_0$ coniugato di se stesso. ---Allora $y_0=x_0$; e
  poiché $x_0 y_0 \equiv (-1)^{n-1} \mod P$, se ne deduce che dovrà
  essere $n$ dispari. Intanto possiamo dunque dire che se
  $x_0>\frac{1}{2}P$ è coniugato di se stesso $\mod P$, applicando ai
  numeri $P,x_0$ l'algoritmo della ricerca del M.C.D., si perviene al
  resto $1$ dopo un numero dispari $n$ di divisioni. ---Siccome poi
  abbiamo trovato che i numeri $y_1,y_2,\ldots,y_{n-1},y_n=1$ sono i
  resti delle divisioni che si ottengono applicando l'algoritmo della
  ricerca del M.C.D. ai numeri $P,y_0$, ne consegue, essendo qui
  $y_0=x_0$, che i resti $y$ coincidono coi corrispondenti $x$ del
  sistema di divisioni $(P,x_0)$.--- La relazione~\eqref{alpha}
  diventa allora:}
%
{}

\begin{equation}
  \tag{\(\alpha'\)}\label{alpha'}
  x_r\cdot x_{n-r} + x_{r+1}\cdot x_{n-r+1} = P
\end{equation}

\colsni{e la~\eqref{beta}}{}

\begin{equation}
  \tag{\(\beta'\)}\label{beta'}
  x_r\cdot x_s - (-1)^{r+s}\cdot x_{n-r+1}\cdot x_{n-s+1} \equiv 0 \mod P
\end{equation}

\colsni{dalla quale, per $r=s$ si trae l'altra:}

\begin{equation}
  \tag{\(\gamma'\)}\label{gamma'}
  x_r^2 - x_{n-r+1}^2 \equiv 0 \mod P.
\end{equation}

\cols{Il resto intermedio del sistema sarà $x_\frac{n+1}{2}$. Ponendo
  nella~\eqref{alpha'} $r=\frac{n+1}{2}$, si avrà:}
% 
{}

\begin{equation}
  \tag{\(\delta'\)}\label{delta'}
  x_{\frac{n+1}{2}}\left(x_{\frac{n-1}{2}} + x_{\frac{n+3}{2}}\right) = P
\end{equation}

\colsni{dunque il resto intermedio è un divisore di $P$.

  Le relazioni $x_i=y_i$ insieme
  alle~\eqref{alpha'},~\eqref{beta'},~\eqref{gamma'},~\eqref{delta'}
  che ne abbiamo dedotte, si traducono nelle seguenti proprietà:

  \guillemotleft{} Dato un numero $P$, sia $x_0>\frac{P}{2}$ una
  radice della congruenza}
% 
{}

\[x^2\equiv 1 \mod P.\]

\cols{Applicando ai numeri $P,x_0$ l'algoritmo della ricerca del
  M.C.D., la successione dei quozienti e quella dei resti godono delle
  seguenti proprietà:}
% 
{}

\cols{Il numero delle divisioni che bisogna eseguire per giungere al
  resto $1$ è sempre dispari.

  Il \ordinalnum{2} quoziente è uguale al penultimo resto diminuito di $1$.

  Fatta astrazione dei due primi quozienti, nella successione dei
  restanti i quozienti equidistanti dagli estremi sono eguali.

  La somma dei prodotti in croce di due resti consecutivi qualunque, e
  dei corrispondenti equidistanti dagli estremi, è uguale a $P$.

  Il resto intermedio è un divisore di $P$.

  La somma o differenza dei prodotti di due resti qualunque, e dei
  corrispondenti equidistanti dagli estremi, è multipla di $P$.

  Una coppia qualunque di resti equidistanti dagli estremi soddisfa
  alla congruenza}
% 
{}

\[x^2-y^2\equiv 0 \mod P.\text{ \guillemotright}\]

\cols{Adunque la conseguenza dei divisori di $P$ e delle soluzioni
  della congruenza $x^2-y^2\equiv 0 \mod P$ dipende dalla conoscenza
  delle radici della congruenza}
% 
{}

\[x^2\equiv 1 \mod P.\]

\cols{---Come secondo caso notevole, supponiamo che $x_0$ soddisfi
  alla congruenza}
% 
{}

\[x_0^3 + 1 \equiv 0 \mod P\]

\colsni{allora}{then}

\[x_0^2 \equiv -1 \mod P.\]

\cols{Ma è pure}{}

\[x_0 y_0\equiv (-1)^{n-1} \mod P\quad \left(
    \begin{array}{c}
      P > x_0 > \frac{P}{2}\\
      P > y_0 > \frac{P}{2}
    \end{array}
  \right)\]

\colsni{se ne trae che dovrá essere $n$ pari ed $y_0=x_0$.

  Per quanto abbiamo osservato, le $y_1,y_2,\ldots,y_{n-1},y_n=1$
  coincidono con le corrispondenti $x_1,x_2,\ldots,x_n=1$; poichè qui
  $(P,x_0)=(P,y_0)$, simboleggiando con $(P,x_0),(P,y_0)$ i due
  sistemi di divisioni che bisogna eseguire per trovare il M.C.D. (il
  quale è $1$) tra i numeri $P,x_0$; e rispettivamente $P,y_0$. Le
  relazioni~\eqref{alpha},~\eqref{beta}}
% 
{}

\begin{align*}
  &x_r\cdot y_{n-r} + x_{r+1}\cdot y_{n-r+1} = P\\
  &x_r\cdot y_s + (-1)^{n+r+s}\cdot x_{n-s+1}\cdot y_{n-r+1} \equiv 0 \mod P
\end{align*}

\colsni{diventano qui, considerando che $n$ è pari:}{}

\begin{align*}
  &x_r\cdot x_{n-r} + x_{r+1}\cdot x_{n-r+1} = P
  \tag{\(\alpha''\)}\label{alpha''}\\
  &x_r\cdot s_s + (-1)^{r+s}\cdot x_{n-r+1}\cdot x_{n-s+1} \equiv 0 \mod P.
    \tag{\(\beta''\)}\label{beta''}
\end{align*}

\cols{Dalla~\eqref{beta''} si ha in particolare per $r=s$}
%
{}

\begin{equation}
  x_r^2 + x_{n-r+1}^2 \equiv 0 \mod P.
  \tag{\(\gamma''\)}\label{gamma''}
\end{equation}

\cols{I resti intermedii sono $x_{\frac{n}{2}}, x_{\frac{n+2}{2}}$;
  ponendo nella~\eqref{alpha''} $r=\frac{n}{2}$, si ha:}
%
{}

\begin{equation}
  x_{\frac{n}{2}}^2 +  x_{\frac{n+2}{2}}^2 = P.
  \tag{\(\delta''\)}\label{delta''}
\end{equation}

\cols{Dunque i resti intermedii soddisfano all'equazione:}
%
{}

\[x^2 + y^2 = P.\]

\cols{Le relazioni $x_i=y_i$ insieme con
  le~\eqref{alpha''},~\eqref{beta''},~\eqref{gamma''},~\eqref{delta''},
  che ne abbiamo dedotte, si traducono nelle seguenti proprietà:

  \guillemotleft{} Dato un numero $P$, sia $x_0>\frac{P}{2}$ una
  radice della congruenza $x^2+1\equiv 0\mod P$, supposta
  solubile. Applicando ai numeri $P,x_0$ l'algoritmo della ricerca del
  M.C.D. si perviene al resto $1$ dopo un numero pari $n$ di
  divisioni.

  La successione dei quozientii e quella dei resti godono le seguentti
  proprietà:

  Il \ordinalnum{2} quoziente è uguale al penultimo resto diminuito di
  $1$.

  Fatta astrazione dai due primi quozienti, nella successione dei
  restanti i quozienti equidistanti dagli estremi sono uguali.

  La somma dei prodotti in croce di due resti consecutivi qualunque e
  dei corrispondenti equidistanti dagli estremi è uguale a $P$.

  La somma o differenza dei prodotti di due resti qualunque e dei
  corrispondenti equidistantii dagli estremi è multipla di $P$.

  Una coppia qualunque di resti equidistanti dagli estremi soddisfa
  alla congruenza  
}

\[x^2 + y^2 \equiv 0 \mod P.\]

\cols{La coppia dei resti intermedii soddisfa all'equazione:}
%
{}

\[x^2 + y^2 = P\]

\cols{Il metodo esposto fa dunque dipendere la conoscenza delle
  soluzioni di quest'ultima equazione dalla conoscenza delle radici
  della congruenza}
%
{}

\[x^2+1\equiv 0 \mod P;\]

\colsni{per ogni radice $x_0$ di questa congrenza si trova una
  soluzione dell'equazione $x^2+y^2=P$; a radici diverse corrispondono
  soluzioni diverse. Operando su tutte le radici $x_0>\frac{P}{2}$
  della congruenza $x^2+1\equiv 0\mod P$ si otterranno altrettante
  soluzioni distinte dell'equazione}
%
{}

\[x^2+y^2=P\]

\colsni{(vedremo più avanti che con questo metodo si ottengono
  \emph{tutte le soluzioni proprie}).

  \begin{corollary}
    Siccome per ogni numero primo $p$ della forma $4k+1$ è sempre
    solubile la congruenza $x^2+1\equiv 0 \mod P$, se ne deduce:

    Ogni numero primo della forma $4k+1$ è sempre scomponibile nella
    somma di due quadrati interi.\footnotemark
  \end{corollary}
}
%
{(we will see later that with this method we obtain \emph{all proper
    solutions}).

  \begin{corollary}
    Since for every prime number $p$ of the form $4k+1$ the congruence
    $x^2+1\equiv 0 \mod P$ always has a solution, we deduce:

    Every prime number of the form $4k+1$ is always decomposoable as a
    sum of two square integers.\footnotemark[\thefootnote]
  \end{corollary}
}
%
\footnotecols{Vedremo più innanzi come questa scomposizione sia
  unica.}{We will see later that this decomposition is unique.}

\secbreak

\cols{Ripigliamo le relazioni}

\begin{align}
  \tag{\ref{1}}
  x_{i-1} &= q_i x_i + x_{i+1}\\
  &&(i=0,1,2,\ldots, n)\notag\\
  \tag{\ref{2}}
  y_{n-i} &= q_i y_{n-i+1} + y_{n-i+2}.
\end{align}

\cols{Si abbia una congruenza della forma:}

\[\sum_{h=0}^n C_h x^{n-h}\cdot y^h \equiv 0 \mod P\]

\colsni{nella quale im primo membro è dunque una frazione (razionale
  intera) omogenea di grado $n$ nelle $x,y$ a coefficienti
  interi. Supponiamo che $x=x_1,y=1$, ne sia una soluzione, cioè che si
  abbia:}
%
{}

\[\sum_{h=0}^n C_h x_1^{n-h} \equiv 0 \mod P\]

\colsni{dico che sussisterà pure la congruenza}{}

\[\sum (-1)^{h(i-1)}\cdot C_h x_i^{n-h}\cdot y_{n-i+1}^h\equiv 0 \mod P.
  \qquad (i=1,2,\ldots,n)\]

\cols{Abbiamo dimostrato che}{}

\[x_i \equiv (-1)^{i-1}x_i\cdot y_{n-i+1} \mod P.\]

\cols{Avremo pertanto}

\begin{equation*}
  \sum (-1)^{h(i-1)}\cdot C_h x_i^{n-h} \cdot y_{n-i+1}^h \equiv
  \sum (-1)^{h(i-1)}\cdot C_h (-1)^{(n-h)(i-1)} \cdot x_1^{n-h} \cdot y_{n-i+1}^h  \equiv
  (-1)^{n(i-1)}\cdot y_{n-i+1}^n \cdot \sum_{h=0}^n C_h x_1^{n-h} \equiv 0 \mod P.
\end{equation*}

\cols{Concludiamo:

  Data una congruenza di grado $n$
  $\sum_{h=0}^nC_h x^{n-h}\equiv 0 \mod P$, ne sia $x_1<\frac{P}{2}$
  una radice. Posto $x+0=P-x_1$, sia $y_0>\frac{P}{2}$ la radice della
  congruenza $x_0y\equiv\pm 1\bmod P$ (che si calcola col procedimento
  sopra esposto): si applichi l'algoritmo della ricerca del
  M.C.D. alle due coppie $(P,x_0)$, $(P,y_0)$ (e basta far ciò per la
  prima coppia): sia $n$ il numero delle divisioni eseguite per
  giungere al resto $1$. Sia $x_1$ \footnotemark{} la serie dei resti
  del sistema $(P,x_0)$; $y_{n-i+1}$ quella del sistema $(P,y_0)$
  $(i=1,2,\dots,n)$.

  Allora
}
%
{}
\footnotetext{$x_i$, maybe?}

\[\sum_{h=0}^n (-1)^{h(i-1)} \cdot C_h x_i^{n-h} \cdot y_{n-i+1}^h \equiv 0 \mod P\]

\colsni{cioè}{}

\[\sum_{h=0}^n (-1)^{h(i-1)}\cdot C_h x_i^{n-h} \cdot y_{n-i+1}^h = k_i \cdot P.\]

\cols{Se uno dei $k_i$ fosse uguale ad $1$, avremmo così ottenuta una
  soluzione dell'equazione}
%
{}

\begin{equation}
  \tag{1}\label{binhomform}
  \sum_{h-0}^n C_h x^{n-h}\cdot y^h = P
\end{equation}

\colsni{il primo membro della quale è una forma binaria omogenea di grado $n$.

  Geometricamente la~\eqref{binhomform} è l'equazione di una curva
  algebrica di ordine $n$ che presenta le seguenti particolarità:

  Le tangenti nei punti all'$\infty$ hanno ciascuna con la curva un
  contatto $(\widetilde{n-1})$ punto e passano tutte per l'origine.

  Ogni retta per l'origine incontra la curva in un solo punto reale,
  se $n$ è dispari, in due soli punti reali se $n$ è pari.

  In quest'ultimo caso l'origine è centro di simmetria per la curva.

  Per vedere se col metodo delle divisioni successive sopra accennato
  si possa ottenere una soluzione della~\eqref{binhomform}, noi
  dovremo fare il cammino inverso, mostrando come da una soluzione di
  essa equazione possa dedursi una radice della congruenza
}
%
{}

\[\sum_{h=0}^n C_h x^{n-h} \equiv 0 \mod P.\]

\cols{Sia dunque $(x_0,y_0)$ una soluzione propria
  della~\eqref{binhomform} (cioè con $x_0,y_0$ primi tra loro):
  supponiamo inoltre che $x_0$ sia primo con $C_n$ ed $y_0$ con $C_0$,
  altrimenti si potrebbe nella~\eqref{binhomform} raccogliere fuori di
  $\sum$ un fattore che dovrebbe dividere anche $P$.

  Di più nella~\eqref{binhomform} i coefficienti $C_h$ potranno essere
  in parte nulli, non però $C_0$ e $C_n$, altrimenti si potrebbe
  raccogliere fuori del segno di sommatoria una delle variabili $x,y$;
  il che noi vogliamo escludere.

  Ciò posto, distinguiamo due casi:

  I. $C_0$ e $C_n$ hanno lo stesso segno: possiamo allora supporre
  senz'altro $C_0$ e $C_n$ entrambi positivi: si divida
  $C_n y_0^{n-1}$ per $x_0$, e sia $q_0$ il quoziente, $x_1$ il resto,
  talché }
%
{}

\[C_n y_0^{n-1} - q_0 x_0 = x_1.\]

\cols{Si divida similmente $C_0 x_0^{n-1}$ per $y_0$, e sia $t_0$ il
  quoziente, $y_1$ il resto, di guisa che:}

\[C_0 x_0^{n-1} - t_0 y_0 = y_1.\]

\cols{Per le ipotesi poste sarà $x_0$ primo con $x_1$, $y_0$ con
  $y_1$. Applichiamo alle due coppie $(x_0,x_1)$, $(y_0,y_1)$
  l'algoritmo della ricerca del M.C.D., e siano}
%
{}

\begin{gather*}
  \left\{
    \begin{array}{l l l l l}
      x_2 & x_3 & \cdots & x_{r-1} & x_r=1\\
      q_2 & q_3 & \cdots & q_{r-1} & q_r
    \end{array}
  \right.\\
  \left\{
    \begin{array}{l l l l l}
      y_2 & y_3 & \cdots & y_{s-1} & y_s=1\\
      t_2 & t_3 & \cdots & t_{s-1} & t_s
    \end{array}
  \right.
\end{gather*}

\colsni{le successioni dei resti e dei quozienti nei sistemi di
  divisioni $(x_0,x_1)$, $(y_0,y_1)$ rispettivamente.

  Si avranno le relazioni}
%
{}

\begin{align}
  \tag{\ref{alpha}}
  x_i - q_{i+2} x_{i+1} &= x_{i+2} & (i=0,1,2,\ldots,r-1)\\
  \tag{\ref{beta}}
  y_j - t_{j+2} y_{j+1} &= y_{j+2} & (j=0,1,2,\ldots,s-1)
\end{align}

\colsni{con l'avvertenza di porre $x_{r+1}=1$\hfill
  $q_{r+1}=x_{r-1}-1$\hfill $y_{s+1}=1$\hfill $t_{s+1}=y_{s-1}-1$.

  Ciò posto, poniamo i due sistemi:}
%
{}

\begin{align}
  \tag{\ref{alpha'}}
  x_{s+1-\rho}' - t_{s+1-\rho}x_{s-\rho}' &= x_{s-\rho-1}'
  &(\rho=0,1,2,\ldots,s)\\
  \tag{\ref{beta'}}
  y_{r+1-\lambda}' - q_{r+1-\lambda}y_{r-\lambda}' &= y_{r-\lambda-1}'
  &(\lambda=0,1,2,\ldots,r)
\end{align}

\colsni{con l'avvertenza di fare $x_0'=x_0$\hfill $x_{-1}'=x_1$\hfill
  $y_0'=y_0$\hfill $y_{-1}'=y_1$}
%
{}

\[q_1 = t_1 = q_0 + t_0 + \sum_{h=1}^{n-1} C_h x_0^{n-h-1} \cdot y_0^{h-1}\]

\colsni{con ciò abbiamo due sistemi lineari, l'uno di $s+1$ equazioni
  ad $s+1$ incognite $x_1', x_2', \ldots, x_{s+1}'$; l'altro di $r+1$
  equazioni ad altrettante incognite $y_1',y_2',\ldots,y_{r+1}'$, le
  quali si possono determinare man mano per via ricorrente. Introdotto
  il simbolo di Gauss, si trova precisamente:}
%
{}

\begin{align}
  \tag{1}\label{contfrac1}
  x_{s+1-\rho}' &= [1, q_{r+1}, q_r, q_{r-1}, \ldots, q_2, t_1, t_2, \ldots, t_{s+1-\rho}]\\
  \tag{2}\label{contfrac2}
  y_{r+1-\lambda}' &= [1, t_{s+1}, t_s, t_{s-1}, \ldots, t_2, q_1, q_2, \ldots, q_{r+1-\lambda}]
\end{align}

\cols{mentre si ha}{}

\begin{align}
  \tag{3}\label{contfrac3}
  x_i &= [1, q_{r+1}, q_r, q_{r-1}, \ldots, q_{i+2}]
  &(i=0,1,2,\ldots,r-1)\\
  \tag{4}\label{contfrac4}
  y_j &= [1, t_{s+1}, t_s, t_{s-1}, \ldots, t_{j+2}].
  &(j=0,1,2,\ldots,s-1)
\end{align}

\cols{Possiamo riunire le $x$ accentate e non accentate, e similmente
  le $y$, attribuendo un unico index variabile da $m=r+s+1$ a zero;
  corrispondentemente si possono denotare con una sola lettera $q$
  affetta da indici, i quozienti $q_i$ e $t_j$; allora le
  relazioni~\eqref{contfrac1},~\eqref{contfrac2}~\eqref{contfrac3},~\eqref{contfrac4}
  possono raccogliersi nelle sole:}
% 
{}

\begin{align}
  \tag{\ref{contfrac1}}
  x_i = [1, q_m, q_{m-1}, q_{m-2}, \ldots, q_{i+1}]\\
  \notag && (i=m-1,m-2,\ldots,2,1,0)\\
  \tag{\ref{contfrac2}}
  y_{m-i-1} = [1, q_1, q_2, \ldots, q_{i+1}]
\end{align}

\colsni{dove adesso $q_m, q_{m-1}, q_{m-2}, \ldots, q_{m-r+1}$
  rappresentano quei quozienti prima denotati con
  $q_{r+1}, q_r, q_{r-1}, q_2$; mentre
  $q_{m-r}, q_{m-r-1}, \ldots, q_1$ rappresentano quei quozienti che
  prima si erano chiamati rispettivamente $t_1, t_2, \ldots,
  t_{s+1}$. ---La soluzione iniziale $(x_0,y_0)$ dell'equazione
  $\sum_{h=0}^n C_h x^{n-h}\cdot y^h=P$ (supposta esistente) è adesso
  rappresentata dalla coppia $(x_{m-r}, y_{r+1})$.

  Ciò posto, formiamo lo specchio:}
%
{}

\[\begin{array}{r@{\;}l l l l l l}
    &x_0 & x_1 & x_2 & \cdots & x_m = 1 & x_{m+1} = 1\\
    1=&y_{m+1} & y_m=1 & y_{m-1} & \cdots & y_1 & y_0.
  \end{array}\]

\cols{Dico che due termini appartenenti alla stessa colonna soddisfano
  alla congruenza}
%
{}

\begin{align*}
  \sum_{h=0}^n (-1)^{hi} C_h x_i^{n-h}\cdot y_{m-i+1}^h \equiv\footnote{\(\equiv 0\)?} \mod P.
  &&(i=0,1,2,\ldots,m)
\end{align*}

\cols{Sia $H$ la radice della congruenza}{}

\[y_{r+1} \equiv x_{m-r}\cdot x \mod P\]

\colsni{sempre solubile, nell'ipotesi che $x_{m-r},y_{r+1}$ siano
  primi fra loro e con $P$.--- Avremo:}

\begin{equation}
  \tag{\(a\)}\label{a}
  y_{r+1} \equiv H x_{m-r} \mod P.
\end{equation}

\cols{Ora dico che sussiste la congruenza}{}

\begin{align}
  \tag{\(b\)}\label{b}
  y_{i+1} - (-1)^{r-i}\cdot H x_{m-i} \equiv 0 \mod P.
  &&(i=0,1,2,\ldots,m)
\end{align}

\cols{Intanto, come mostra la~\eqref{a}, la~\eqref{b} è vera per
  $i=r$. Facciamo ora vedere che la stessa congruenza sussiste anche
  per $i=r-1$ e per $i=r+1$. --Ricordiamo che $x_m,y_{r+1}$ sono quei
  numeri che prima avevamo chiamato $x_0,y_0$; e che
  $x_{m-r-1},y_{r+2},x_{m-r+1},y_r$ rappresentano quelle quantità che
  prima si erano denotate con $x_1', y_1,x_1,y_1'$
  rispettivamente. ---Talché per ipotesi e per costruzione avremo:}
%
{}

\begin{align*}
  x_{m-r+1} &= C_n y_{r+1}^{n-1} - q_0 x_{m-r}
              \qquad x_{r+1} = C_0 x_{m-r}^{n-1} - t_0 y_{r+1}\\
  x_{m-r-1} &= \left[ q_0 t_0 + \sum_{h=1}^{n-1} C_h x_{m-r}^{n-h-1} \cdot y_{r-1}^{h-1} \right]
              \times x_{m-r} + C_n y_{r+1}^{n-1} - q_0 x_{m-r} =
              t_0 x_{m-r} + \sum_{h=0}^{n-1} C_h x_{m-r}^{n-h} \cdot y_{r+1}^{h-1}
\end{align*}
%
\begin{equation*}
  y_r = \left[ q_0 + t_0 + \sum_{h=1}^{n-1} C_h x_{m-r}^{n-h-1} \cdot y_{r+1}^{h-1}\right]
  \times y_{r+1} + C_0 x_{m-r}^{n-1} - t_0 y_{r+1} =
  q_0 y_{r+1} + \sum_{h=0}^{n-1} C_h x_{m-r}^{n-h-1}\cdot y_{r+1}^h
\end{equation*}
%
\begin{equation*}
  \sum_{h=0}^n C_h x_{m-r}^{n-h}\cdot y_{r+1}^h = P.
\end{equation*}

\cols{Avremo perciò}{}

\begin{equation*}
  y_r + H x_{m-r+1} = q_0 y_{r+1} + \frac{P - C_n y_{r+1}^n}{x_{m-r}}
  + H (C_n y_{r+1}^{n-1} - q_0 y_{m-r})
\end{equation*}
%
\begin{multline*}
  (y_r + H x_{m-r+1}) x_{m-r} = q_0 y_{r+1} x_{m-r} + P - C_n y_{r+1}^n
  + H x_{m-r} (C_n y_{r+1}^{n-1} - q_0 x_{m-r}) \equiv\\
  \equiv q_0 x_{m-r} (y_{r+1} - H x_{m-r}) - C_n y_{r+1}^{n-1}(y_{r+1} - H x_{m-r}) \equiv
  (q_0 x_{m-r} - C_n y_{r+1}^{n-1}) \times (y_{r+1} - H x_{m-r}) \equiv 0 \mod P
\end{multline*}

\colsni{ed essendo $x_{m-r}$ primo con $P$, se ne deduce}
%
{}

\[y_r + H x_{m-r+1} \equiv 0 \mod P.\]

\cols{Similmente}{Similarly}

\begin{equation*}
  y_{r+2} + H x_{m-r-1} = t_0 (y_{r+1} - H x_{m-r}) + C_0 x_{m-r}^{n-1}
  + H \frac{P - C_0 x_{m-r}^n}{y_{r+1}} \equiv
  C_0 x_{m-r}^{n-1} + H\frac{P - C_0 x_{m-r}^n}{y_{r+1}} \mod P
\end{equation*}
%
\begin{equation*}
  (y_{r+2} + H x_{m-r-1}) y_{r+1} \equiv C_0 x_{m-r}^{n-1} y_{r+1} - C_0 x_{m-r}^n \cdot H
  \equiv C_0 x_{m-r}^{n-1}(y_{r+1} - H x_{m-r}) \equiv 0 \mod P
\end{equation*}

\end{document}

% Local Variables:
% ispell-local-dictionary: "italian"
% End:
