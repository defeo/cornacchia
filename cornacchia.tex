\documentclass[12pt]{article}

\usepackage[a4paper,margin=1in,landscape]{geometry}
\usepackage[utf8]{inputenc}
\usepackage{parcolumns}
\usepackage[italian,american]{babel}
\usepackage{amsmath,amssymb,amsthm}

\newcommand{\cols}[2]{\begin{parcolumns}[rulebetween]{2}%
    \selectlanguage{italian}\colchunk{#1}%
    \selectlanguage{american}\colchunk{#2}%
  \end{parcolumns}}
\newcommand{\colsni}[2]{\begin{parcolumns}[rulebetween,nofirstindent]{2}%
    \selectlanguage{italian}\colchunk{#1}%
    \selectlanguage{american}\colchunk{#2}%
  \end{parcolumns}}


\title{Su di un metodo per la risoluzione in numeri interi
  dell'equazione\\
  $\sum_{h=0}^nC_hx^{n-h}\cdot y^h=P$}
\author{Dott. Giuseppe Cornacchia\\[1em]
  translation by Luca De Feo}

\begin{document}
\maketitle

\cols{Il metodo che andiamo ad esporre si fonda sul noto
  \emph{algoritmo di Eulero} che si applica alla risoluzione
  dell'equazione $ax+by=1$.
  
  Se si hanno due quantità indeterminate qualunque $a,b,$ e una serie
  di altre quantità $\gamma,\delta,\varepsilon,\dots,\lambda,\mu,\nu$,
  e si forma con esse una nuova serie di quantità $c,d,e,\dots,l,m,n$
  operando secondo la legge seguente:}
{The method we're going to explain is funded on Euler's
  well known algorithm, which applies to the solution of the
  equation $ax+by=1$.
  
  If we have two arbitrary indeterminate quantities $a,b,$ and a
  series of other quantities
  $\gamma,\delta,\varepsilon,\dots,\lambda,\mu,\nu$, and we form with
  them a new series of quantities $c,d,e,\dots,l,m,n$ acting by the
  following law:}

\[c = \gamma b + a\quad
  d = \delta c + b\quad
  e = \varepsilon d + e,\dots,
  n = \nu m + l\]

\colsni{si potranno ottenere tutte le quantità $c,d,e,\dots,l,m,n,$
  successivamente, in funzione di $a$ e $b$. In particolare:}
%
{we can obtain all quantities  $c,d,e,\dots,l,m,n,$
  successively, as a function of $a$ e $b$. In particular:}

\[n = Ga+Hb\]

\colsni{dove $G,H$ sono funzioni di $\gamma,\delta,\varepsilon,\dots$
  indipendenti da $a$ e $b$. Usando il simbolo di Gauss, si ha}
%
{where $G,H,$ are function of $\gamma,\delta,\varepsilon,\dots$
  independent from $a$ and $b$. Using Gauss' symbol, we have}

\[H = [\gamma,\delta,\varepsilon,\dots,\lambda,\mu,\nu]\quad
  G=[\delta,\varepsilon,\dots,\lambda,\mu,\nu].\]

\cols{Le espressioni $H$ obbediscono a due semplici leggi di
  formazione, l'una procedendo verso destra, l'altra verso sinistra,
  che sono espresse dalle formule:}
%
{The expressions $H$ obey two simple laws, one proceding from the
  right, the other from the left, as expressed by the formulas:}
%
\begin{align*}
  [\gamma,\delta,\varepsilon,\dots,\lambda,\mu,\nu] &=
  \gamma[\delta,\varepsilon,\dots,\lambda,\mu,\nu] +
  [\varepsilon,\dots,\lambda,\mu,\nu]\\
  [\gamma,\delta,\varepsilon,\dots,\lambda,\mu,\nu] &=
  [\gamma,\delta,\varepsilon,\dots,\lambda,\mu]\nu +
  [\gamma,\delta,\varepsilon,\dots,\lambda].
\end{align*}

\end{document}