\documentclass[12pt,leqno]{article}

\usepackage[a4paper,margin=1in,landscape]{geometry}
\usepackage[utf8]{inputenc}
\usepackage[T1]{fontenc}
\usepackage{parcolumns}
\usepackage[italian,american]{babel}
\usepackage{amsmath,amssymb,amsthm}
\usepackage{fmtcount}

\newcommand{\cols}[2]{\begin{parcolumns}[rulebetween]{2}%
    \selectlanguage{italian}\colchunk{#1}%
    \selectlanguage{american}\colchunk{#2}%
  \end{parcolumns}}
\newcommand{\colsni}[2]{\begin{parcolumns}[rulebetween,nofirstindent]{2}%
    \selectlanguage{italian}\colchunk{#1}%
    \selectlanguage{american}\colchunk{#2}%
  \end{parcolumns}}

\title{Su di un metodo per la risoluzione in numeri interi
  dell'equazione\\
  $\sum_{h=0}^nC_hx^{n-h}\cdot y^h=P$}
\author{Dott. Giuseppe Cornacchia\\[1em]
  translation by Luca De Feo}

\begin{document}
\maketitle

\cols{Il metodo che andiamo ad esporre si fonda sul noto
  \emph{algoritmo di Eulero} che si applica alla risoluzione
  dell'equazione $ax+by=1$.
  
  Se si hanno due quantità indeterminate qualunque $a,b,$ e una serie
  di altre quantità $\gamma,\delta,\varepsilon,\ldots,\lambda,\mu,\nu$,
  e si forma con esse una nuova serie di quantità $c,d,e,\ldots,l,m,n$
  operando secondo la legge seguente:}
{The method we're going to explain is funded on Euler's
  well known algorithm, which applies to the solution of the
  equation $ax+by=1$.
  
  If we have two arbitrary indeterminate quantities $a,b,$ and a
  series of other quantities
  $\gamma,\delta,\varepsilon,\ldots,\lambda,\mu,\nu$, and we form with
  them a new series of quantities $c,d,e,\ldots,l,m,n$ acting by the
  following law:}

\[c = \gamma b + a\quad
  d = \delta c + b\quad
  e = \varepsilon d + e,\ldots,
  n = \nu m + l\]

\colsni{si potranno ottenere tutte le quantità $c,d,e,\ldots,l,m,n,$
  successivamente, in funzione di $a$ e $b$. In particolare:}
%
{we can obtain all quantities  $c,d,e,\ldots,l,m,n,$
  successively, as a function of $a$ e $b$. In particular:}

\[n = Ga+Hb\]

\colsni{dove $G,H$ sono funzioni di $\gamma,\delta,\varepsilon,\ldots$
  indipendenti da $a$ e $b$. Usando il simbolo di Gauss, si ha}
%
{where $G,H,$ are function of $\gamma,\delta,\varepsilon,\ldots$
  independent from $a$ and $b$. Using Gauss' symbol, we have}

\[H = [\gamma,\delta,\varepsilon,\ldots,\lambda,\mu,\nu]\quad
  G=[\delta,\varepsilon,\ldots,\lambda,\mu,\nu].\]

\cols{Le espressioni $H$ obbediscono a due semplici leggi di
  formazione, l'una procedendo verso destra, l'altra verso sinistra,
  che sono espresse dalle formule:}
%
{The expressions $H$ obey two simple laws, one proceding from the
  right, the other from the left, as expressed by the formulas:}
%
\begin{align*}
  [\gamma,\delta,\varepsilon,\ldots,\lambda,\mu,\nu] &=
  \gamma[\delta,\varepsilon,\ldots,\lambda,\mu,\nu] +
  [\varepsilon,\ldots,\lambda,\mu,\nu]\\
  [\gamma,\delta,\varepsilon,\ldots,\lambda,\mu,\nu] &=
  [\gamma,\delta,\varepsilon,\ldots,\lambda,\mu]\nu +
  [\gamma,\delta,\varepsilon,\ldots,\lambda].
\end{align*}

\cols{Si ha poi}{Then we have}

\[[\gamma,\delta,\varepsilon,\ldots,\lambda,\mu,\nu] =
  [\nu,\mu,\lambda,\ldots,\varepsilon,\delta,\gamma];\quad
  [-\gamma,-\delta,-\varepsilon,\ldots,-\lambda,-\mu,-\nu] =
  \pm[\gamma,\delta,\varepsilon,\ldots,\lambda,\mu,\nu].\]

\cols{Queste espressioni sono della massima importanza nella teoria
  delle frazioni continue. Infatti, se una frazione continua
  ordinaria, i cui numeratori sono tutti uguali ad 1, si rappresenta
  col simbolo
  $(\gamma,\delta,\varepsilon,\ldots,\lambda,\mu,\nu,\ldots)$ talché
  sia}
%
{}

\[(\gamma,\delta,\varepsilon,\ldots,\lambda,\mu,\nu) =
  \gamma+\frac{1}{(\gamma,\delta,\varepsilon,\ldots,\lambda,\mu,\nu)} =
  \left(\gamma,\delta,\varepsilon,\ldots,\lambda,\mu+\frac{1}{\nu}\right)\]

\colsni{si ottiene in generale, mediante riduzione:}{}

\[(\gamma,\delta,\varepsilon,\ldots,\lambda,\mu,\nu) =
  \frac{[\gamma,\delta,\varepsilon,\ldots,\lambda,\mu,\nu]}
  {[\delta,\varepsilon,\ldots,\lambda,\mu,\nu]}.\]

\cols{Se i numeri $\gamma,\delta,\varepsilon,\ldots,\lambda,\mu,\nu$
  sono interi, altrettanto può dirsi dei numeratori e dei denominatori
  delle frazioni}
%
{}

\[\frac{[\gamma]}{[1]}, \frac{[\gamma,\delta]}{[\delta]},
  \frac{[\gamma,\delta,\varepsilon]}{[\gamma,\delta]}, \ldots,
  \frac{[\gamma,\delta,\varepsilon,\ldots,\lambda,\mu,\nu]}
  {[\delta,\ldots,\lambda,\mu,\nu]},\]

\colsni{ognuna delle quali è inoltre irreducibile.

  Vediamo qualche altra propriétà delle espressioni sopra
  accennate. ---Se, prese delle quantità $1,q_1,q_2,\ldots,q_{n-1},q_n$
  si formano successivamente le espressioni:}
%
{}
%
\begin{gather*}
  y_{n+1}=1\quad y_n=1\quad y_{n-1}=[1,q_1]\quad y_{n-2}=[1,q_1,q_2),\ldots\\
  y_1=[1,q_1,q_3,\ldots,q_{n-1}]\quad;\quad
  y_0=[1,q_1,q_2,\ldots,q_{n-1},q_n]
\end{gather*}

\colsni{e poi le epressioni}{}
%
\begin{gather*}
  x_{n+1}=1\quad x_n=1\quad x_{n-1}=[1,q_n]\quad x_{n-2}=[1,q_n,q_{n-1}],\ldots\\
  x_1=[1,q_n,q_{n-1},\ldots,q_2]\quad;\quad
  x_0=[1,q_n,q_{n-1},\ldots,q_2,q_1]  
\end{gather*}

\colsni{dico che si ha:}{}
%
\begin{align}
  &x_r\cdot y_{n-r} + x_{r+1}\cdot y_{n-r+1} = P
  &(r=0,1,2,\dots,n)
  \tag{\(\alpha\)}\label{1.a}\\
  &x_r\cdot y_s + (-1)^{n+r+s}\cdot x_{n-s+1}\cdot y_{n-r+1} \equiv 0 \mod P
  &\left(\begin{array}{c}r=0,1,\dots,n\\s=0,1,\dots,n\end{array}\right)
  \tag{\(\beta\)}\label{1.b}
\end{align}

\colsni{avendo posto $P=x_0+x_1=y_0+y_1$. La \eqref{1.b} vale
  naturalmente solo quando le $q$ sono numeri interi. ---Per la legge
  esposta di formazione, si ha:}
%
{}

\[x_{r+1} = x_{r-1} - q_r x_r \quad y_{n-r}=q_r y_{n-r+1} + y_{n-r+2}\]

\colsni{posto}{}

\[q_0 = 1 \quad x_{-1} = x_0 + x_1 = P.\]

\cols{Sostituendo nel \ordinalnum{1} membro della \eqref{1.a},
  avremo:}{}

\[x_r\cdot y_{n-r} + x_{r+1}\cdot y_{n-r+1} =
  x_r(q_r y_{n-r+1} + y_{n-r+1}) + y_{n-r+1}(x_{r-1} - q_r x_r) =
  x_{r-1}\cdot y_{n-r+1} + x_r\cdot y_{n-r+2}\]

\cols{Dunque l'espressione $x_r\cdot y_{n-r} + x_{r+1}\cdot y_{n-r+1}$
  non cambia valore mutando $r$ in $r-1$, e quindi neppure cambiando
  $r$ in $r-\lambda (\lambda=0,1,2,\dots r)$.  Laonde:}
%
{}

\[x_r\cdot y_{n-r} + x_{r+1}\cdot y_{n-r+1} =
  x_{r-\lambda}\cdot y_{n-(r-\lambda)} + x_{(r-\lambda)+1}\cdot y_{n-(r-\lambda)+1}.\]

\colsni{e per $r=\lambda$ si ha}{}

\[x_r\cdot y_{n-r} + x_{r+1}\cdot y_{n-r+1} = x_0 + x_1 = y_0 + y_1 = P.\]

\cols{Così è dimostrata la \eqref{1.a}; dalla \eqref{1.b} si possono
  come casi particolari estrarre le due congruenze}
%
{}
%
\begin{align}
  x_r + (-1)^rx_1\cdot y_{n-r+1} &\equiv 0 \mod P
  \tag{\(\beta_1\)}\label{2.a}\\
  y_s + (-1)^s\cdot y_1 \cdot x_{n-s+1} &\equiv 0 \mod P
  \tag{\(\beta_2\)}\label{2.b}
\end{align}
 
\colsni{le quali intanto sussistono, come si verifica subito, la
  \ordinalnum{1}[f] per $r=0,1$; la \ordinalnum{2}[f] per
  $s=0,1$. ---E col solito metodo da $r$ a $r+1$ (o da $s$ ad $s+1$)
  valendosi delle relazioni che legano le quantità $x$ e le $y$, si
  dimostra che sono vere per qualunque valore degli indici $r,s$
  rispettivamente.--- Le \eqref{2.a}, \eqref{2.b} si possono anche
  scrivere:}
%
{}
%
\begin{align}
  x_r &\equiv - (-1)^rx_1\cdot y_{n-r-1} \mod P
  \tag{\(\beta_1'\)}\label{3.a}\\
  y_s &\equiv - (-1)^sy_1\cdot x_{n-s-1} \mod P
  \tag{\(\beta_2'\)}\label{3.b}
\end{align}

\colsni{e moltiplicando membro a membro}{}

\[x_r\cdot y_s \equiv (-1)^{r+s}\cdot x_1\cdot y_1\cdot y_{n-r+1}\cdot x_{n-s+1}.\]

\cols{Ma la \ref{3.a} dà per $r=n$}{}

\[1 \equiv - (-1)^n x_1 \cdot y_1\]

\colsni{e però, come si voleva:}

\[x_r\cdot y_s + (-1)^{n+r+s}\cdot x_{n-s+1}\cdot y_{n-r+1} \equiv 0 \mod P.\]

\cols{Ciò posto, sia $P$ un intero qualunque, ed $a$ un numero primo
  con $P$ e minore di $P$ ma maggiore di $\frac{1}{2}P$. Si applichi
  alla coppia $(P,a)$ l'algoritmo della ricerca del M.C.D.; per
  l'ipotesi posta si perverrà al resto $1$ dopo un numero finito $n$
  di divisioni. Sia $q_0=1,q_1,q_2,\ldots,q_{n-1}$ la serie dei
  quozienti; $x_1,x_2,\ldots,x_n=1$ la serie dei resti corrispondenti.

  Posto $P=x_{-1},a=x_0,x_{n+1}=1,q_n=x_{n-1}-1$ avremo:}
%
{}

\begin{equation}
  \label{1}
  x_{i-1} = q_i x_i + x_{i+1}.\qquad
  (i=0,1,\ldots, n)
\end{equation}

\cols{Si ponga ora il sistema}{}

\begin{equation}
  \label{2}
  y_{n-i} = q_i y_{n-i+1} + y_{n-i+2}\qquad
  (i=0,1,\ldots, n)
\end{equation}

\colsni{a cui si aggiungano le condizioni $y_n=y_{n+1}=1$.

  Si avrà allora, introducendo il simbolo di Gauss:}

\[y_{n-i} = [1,q_1,q_2,\ldots,q_i]\qquad
  x_{i-1} = [1,q_n,q_{n-1},\ldots,q_i].\]

\cols{Richiamando le propriétà sopra dimostrate, avremo:}{}

\begin{align}
  &x_r\cdot y_{n-r} + x_{r+1}\cdot y_{n-r+1} = x_0 + x_1 = y_0 + y_1 = P
  &(r=0,1,2,\dots,n)
  \tag{\ref{1.a}}\\
  &x_r\cdot y_s + (-1)^{n+r+s}\cdot x_{n-s+1}\cdot y_{n-r+1} \equiv 0 \mod P
  &\left(\begin{array}{c}r=0,1,\dots,n\\s=0,1,\dots,n\end{array}\right)
  \tag{\ref{1.b}}
\end{align}

\colsni{da cui in particolare per $r=s=1$}{}

\[x_1\cdot y_1 \equiv (-1)^{n-1} \mod P\]

\colsni{e ancora}{and}

\[x_0\cdot y_0 \equiv (-1)^{n-1} \mod P.\]

\cols{Si è trovata una soluzione $y=y$ della congruenza}{}

\[ay \equiv \pm 1 \mod P.\]

\cols{Ponendo mente alle relazioni ricorrenti~\eqref{2} e ricordando
  che $y_0+y_1=P$, si scorge come i numeri $y_1,y_2,\dots,y_n=1$ sono
  i resti delle successive divisioni che si ottengono applicando ai
  numeri $(P,y_0)$ l'algoritmo della ricerca del M.C.D. Si ha poi
  $x_0>\frac{1}{2}P,y_0>\frac{1}{2}P$. Se $n$ è dispari, allora
  $x_0 y_0 \equiv 1\mod P$, cioè la radice della congruenza}
%
{}

\[x_0 y \equiv 1 \mod P\]

\colsni{è maggiore di $\frac{1}{2}P$; se $n$ è pari, si ha
  $x_0y\equiv -1, x_0 y_1 \equiv 1 \mod P$; in questo caso la radice
  $y_1$ della congruenza $x_0y\equiv 1 \mod P$ è $<\frac{1}{2}P$. Dopo
  ciò possiamo enunciare:

  \guillemotleft Dato un intero qualunque $P$ e un numero
  $x_0<\frac{1}{2}P$, primo con $P$ e inferiore ad esso, sia $y_0$ la
  radice della congruenza}
%
{}

\[x_0 y \equiv 1\mod P\]

\colsni{in altri termini, sia $y_0$ il \emph{coniugato} di
  $x_0\mod P$: applicando ai numeri $(P,x_0)$ l'algoritmo della
  ricerca del M.C.D. si perviene al resto $1$ dopo un numero $n$ di
  divisioni pari o dispari, secondo che
  $y_0\lessgtr\frac{1}{2}P$. ---In ogni caso, applicando lo stesso
  procedimento ai numeri $(P,y_0)$
  $\left\{ \text{oppure } (P,P-y_0) \text{ se } y_0<\frac{1}{2}P
  \right\}$ si perviene al resto $1$ dopo lo stesso numero $n$ di
  divisioni.--- Le due successioni dei resti e dei quotienti godono le
  seguenti proprietà:

  Nei due sistemi di divisioni il \ordinalnum{2} quoziente dell'un
  sistema è uguale al penultimo resto dell'altro, diminuito di 1. ---I
  quozienti della \ordinalnum{1}[f] successione, a cominciare dal
  terzo, si riproducono nella \ordinalnum{2}[f] in ordine inverso.

  La somma dei prodotti in croce di due resti consecutivi del
  \ordinalnum{1} sistema, e dei corrispondenti nel \ordinalnum{2}
  occupanti i posti complementari ad $n+1$ (cioè se i primi due resti
  occupano i posti $t,t+1$, i corrispondenti occuperanno i posti
  $n-t,n-t+1$) è uguale a $P$.

  Presi due resti qualunque $x_r,y_s$, l'uno del \ordinalnum{1}
  sistema, l'altro del \ordinalnum{2}, e i loro rispettivi
  complementari $x_{n-s+1},y_{n-r+1}$, si ha}
%
{}

\[x_r\cdot y_s + (-1)^{n+r+s}\cdot x_{n-s+1}\cdot y_{n-r+1}\equiv 0 \mod P.\]

\cols{In particolare, presi due resti $x_r,x_{n-r+1}$ del
  \ordinalnum{1} sistema equidistanti dagli estremi, e i
  corrispondenti $y_r,y_{n-r+1}$ nel \ordinalnum{2}, si ha:}
%
{}

\[x_r\cdot y_r + (-1)^n x_{n-r+1}\cdot y_{n-r+1}\equiv 0 \mod P.\]
  
\end{document}
